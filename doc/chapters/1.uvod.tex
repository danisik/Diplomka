%CHAPTER
\chapter{Úvod}
Patenty jsou v dnešní době nedílnou součástí zákonné ochrany nových technologií a vynálezů v průmyslu a jejich odvětví. Vlastník takového patentu má výhradní právo využívat vynález a poskytovat souhlas o využití patentu jiným stranám. Patenty jsou registrovány a spravovány jak v rámci jednotlivých zemí (národní patentové instituce), tak i mezinárodně (například Evropský patentový úřad \gls{EPO}\footnote{\href{https://www.epo.org/}{https://www.epo.org/}}). Tyto instituce veřejně poskytují informace o registrovaných patentech pomocí vyhledávačů na jejich webových stránkách, a v některých případech dokonce i plný export jejich databáze. Pro vědce a vynálezce jsou tyto vyhledávače velice důležité právě proto, aby si mohli zkontrolovat, zda jejich vynález je jedinečný, nebo už ve světě existuje v lehce upravené podobě a je patentován.
\newline
\indent Díky tomu, že patentové instituce poskytují svá data veřejnosti, si může kdokoliv vytvořit vlastní lokální úložiště patentů. Výhoda lokálního úložiště spočívá v tom, že si ho můžeme vytvořit na míru tak, aby bylo co nejrobustnější a nejspolehlivější. Lze si zvolit vlastní infrastrukturu, technologie a hardware vzhledem k tomu, jak bude úložiště využíváno a jaká bude struktura ukládaných dat.
\newline
\indent Cílem této práce je se seznámit s dostupnými zdroji dat o patentech a vytvořit rozsáhlá lokální úložiště patentových dat umožňující jejich efektivní vytěžování. Zdroje dat musí poskytovat své databáze patentů (žádosti i publikace) zdarma a patenty musí obsahovat předem stanovené povinné atributy, aby je bylo možné použít. Získaná data budou následně uložena do specifického typu databáze, která bude umožňovat co nejefektivnější vytěžování uložených dat. To znamená rychlé vyhledávání správných výsledků v relativně krátkém čase pro miliony (až desítky milionů) záznamů. Stažená data budou filtrována na základě specifikovaných atributů, které musí obsahovat (protože ve struktuře existuje element pro povinný atribut, ještě neznamená, že pro něj existuje reálná hodnota).