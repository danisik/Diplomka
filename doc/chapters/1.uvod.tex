%CHAPTER
\chapter{Úvod}
První odstavec by byl o patentu. Jednoduše by se popsalo proč vlastně je potřeba patent, proč ho chce zadávající, lehce popsat co to je atd.
\newline
\indent Druhý odstavec by byl o databázi, lehký popis jako k čemu to je, jak je to např. rozšířený atp.
\newline
\indent Cílem této práce je se seznámit s dostupnými zdroji dat o patentech a vytvořit rozsáhlá lokální úložiště patentových dat umožňující jejich efektivní vytěžování. Zdroje dat musí poskytovat své databáze patentů (žádosti i publikace) zdarma a patenty musí obsahovat předem stanovené povinné atributy, aby je bylo možné použít. Získaná data budou následně uložena do specifického typu databáze, která bude umožňovat co nejefektivnější vytěžování uložených dat. To znamená rychlé vyhledávání správných výsledků v relativně krátkém čase pro miliony (až desítky milionů) záznamů. Import dat bude řešen pomocí jednoduché aplikace, která bude procházet všechna data a filtrovat ty patenty, kterým chybí některé povinné údaje (i přes to, že struktura obsahuje elementy, ve kterých se údaj má nacházet), nebo jsou nevalidní. 

% pomocí ní čtenář získá první dojem z práce. Měla by obsahovat stručný nástin problematiky (jeden až dva odstavce) a pak jasné vysvětlení, co budu dělat a proč. Po přečtení by čtenář měl nabýt dojmu, že je mu jasné, proč byla zadaná práce řešena, co bude jejím obsahem a cílem. Úvod by neměl přesáhnout jednu stranu textu.
% Nepište sem, co jste udělali – to patří do závěru