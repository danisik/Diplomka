\chapter{Návrh úložiště}

%This section provides a general overview of the state of today’s bibliographic
%databases and describes different types of bibliographical databases. First,
%we take a look at publication databases and show some examples of how they
%compare to each other. Next, we describe databases storing bibliographic
%information about patent data and provide some concrete examples of these
%data sources.

Zdroje (které ano, které ne + proč) + udělat stejnou tabulku jak v excelu + zmínit stránku ze který jsem čerpal informace (wipo.int)\todo{todo}\newline
Nezapomenout zmínit meze let, ve kterých se jednotlivé patenty z daných zemí nachází \todo{todo}

\section{Výběr patentů}
Při výběru patentů byly stanoveny tři podmínky, které museli být splněny:
\begin{itemize}
\item Dostupnost - patenty musí být dostupné z online stránek / databází bez poplatků.
\item Datum - patentová přihláška nebo publikace patentu musí být podána alespoň v roce 2000, všechny ostatní patenty budou vyfiltrovány.
\item Atributy - všechny patenty musí obsahovat povinné atributy (viz kapitola č. \ref{subsec:atributy}).
\end{itemize}

\subsection{Zdroje dat}
V dnešním světě existuje několik desítek až stovek patentových zdrojů dat, od webových vyhledávačů v databázi až po plný export databáze s patenty. Velké organizace, jako například \gls{EPO}, \gls{WIPO}, \gls{USPTO}, udržují jedny z největších patentových databází (desítky až stovky milionů patentů), ve kterých lze vyhledávat velké množství informací zdarma za použití webových vyhledávačů na dané stránce organizace. Lze zde najít všechny typy patentů (přihlášky, publikace), národní patenty i patenty registrované například u \gls{EPO}. V případě exportu databází, \gls{USPTO} poskytuje plný export svých databází veřejnosti pro libovolné používání, zcela zdarma. Využití těchto zdrojů dat by bylo určitě skvělé, ale tyto zdroje byly nedávno použity a rozebrány v jiné diplomové práci, proto je vhodné se spíše zaměřit na národní zdroje dat patentů.
\newline
\indent Národní databáze patentů dané země obsahuje všechny národní patenty, některé dokonce i patenty z jiných zemí registrovaných u \gls{EPO}. 
	\begin{table}[H]
	\centering
	\begin{tabular}{|>{\centering\arraybackslash}p{2.2cm}|>{\centering\arraybackslash}p{8cm}|>{\centering\arraybackslash}p{2cm}|} 
	\hline
	\textbf{Země}    & \textbf{Patentový úřad} & \textbf{Zkratka}                \\ 
	\hline
	Anglie & \href{https://www.gov.uk/topic/intellectual-property}{Intellectual Property Office}  & IPO         \\ 
	\hline
	Arménie & \href{https://www.aipa.am/hy/}{Intellectual Property Office}  & -         \\ 
	\hline
	Austrálie & \href{https://www.ipaustralia.gov.au/}{IP Australia}  & -         \\ 
	\hline
	Bělorusko & \href{https://www.ncip.by/}{National Center of Intellectual Property}  & NCIP         \\ 
	\hline
	Bulharsko & \href{https://www.bpo.bg/}{Patent Office of Republic of Bulgaria}  & -         \\ 
	\hline
	Česko & \href{https://upv.gov.cz/}{Industrial Property Office of the Czech Republic}  & -         \\ 
	\hline
	Čína & \href{https://www.cnipa.gov.cn/}{China National Intellectual Property Administration}  & CNIPA         \\ 
	\hline
	Dánsko & \href{https://www.dkpto.org/}{Danish Patent and Trademark Office}  & -         \\ 
	\hline
	Egypt & \href{http://www.egypo.gov.eg}{Egyptian Patent Office}  & -         \\ 
	\hline
	Estonsko & \href{https://www.epa.ee/et}{The Estonian Patent Office}  & -         \\ 
	\hline
	Filipíny & \href{http://www.ipophil.gov.ph/}{Intellectual Property Office of the Philippines}  & IPOPHL         \\ 
	\hline
	Finsko & \href{http://www.prh.fi/en/index.html}{Finnish Patent and Registration Office}  & PRH         \\ 
	\hline
	Francie & \href{http://www.inpi.fr/}{National Institute of Industrial Property}  & INPI         \\ 
	\hline
	Hong Kong & \href{https://www.ipd.gov.hk/index.htm}{Intellectual Property Department}  & -         \\ 
	\hline
	Chorvatsko & \href{https://www.dziv.hr/}{State Intellectual Property Office of the Republic of Croatia}  & SIPO         \\ 
	\hline
	Indie & \href{http://www.ipindia.nic.in/}{Office of the Controller General of Patents, Designs and Trade Marks}  & -         \\ 
	\hline
	Indonésie & \href{http://www.dgip.go.id/}{Directorate General of Intellectual Property}  & DGIP         \\ 
	\hline
	Irsko & \href{https://www.ipoi.gov.ie/en/}{Intellectual Property Office of Ireland}  & IPOI         \\ 
	\hline
	Island & \href{https://www.isipo.is/}{Icelandic Intellectual Property Office}  & ISIPO         \\ 
	\hline
	Israel & \href{https://www.gov.il/en/departments/ilpo}{The Israel Patent Office}  & ILPO         \\ 
	\hline
	Itálie & \href{https://uibm.mise.gov.it/index.php/it/}{Directorate General for the Protection of Industrial Property}  & -         \\ 
	\hline
	Japonsko & \href{https://www.jpo.go.jp/e/index.html}{Japan Patent Office}  & JPO         \\ 
	\hline
	Jižní Korea & \href{http://www.kipo.go.kr/}{Korean Intellectual Property Office}  & KIPO         \\ 
	\hline	
	Kanada & \href{https://www.ic.gc.ca/}{Canadian Intellectual Property Office}  & CIPO         \\ 
	\hline
	\end{tabular}
	\caption{Národní patentové úřady a jejich zkratky, část první}
	\label{tab:table_offices1}
	\end{table}

\newpage

	\begin{table}[H]
	\centering
	\begin{tabular}{|>{\centering\arraybackslash}p{2.2cm}|>{\centering\arraybackslash}p{8cm}|>{\centering\arraybackslash}p{2cm}|} 
	\hline
	\textbf{Země}    & \textbf{Patentový úřad} & \textbf{Zkratka}                \\ 
	\hline
	Kuba & \href{http://www.ocpi.cu}{Cuban Industrial Property Office}  & OCPI         \\ 
	\hline
	Litva & \href{http://vpb.lrv.lt/en/}{State Patent Bureau of the Republic of Lithuania}  & -         \\
	\hline
 	Lotyšsko & \href{https://www.lrpv.gov.lv/lv}{Patent Office of the Republic of Latvia}  & -         \\ 
	\hline
	Maďarsko & \href{http://www.hipo.gov.hu/}{Hungarian Intellectual Property Office}  & HIPO         \\ 
	\hline
	Malajsie & \href{http://www.myipo.gov.my/}{Intellectual Property Corporation of Malaysia}  & MyIPO         \\ 
	\hline
	Mexiko & \href{https://www.gob.mx/impi/en}{Instituto Mexicano De La Propiedad Industrial}  & IMPI         \\ 
	\hline
	Moldova & \href{http://www.agepi.gov.md/}{State Agency on Intellectual Property}  & AGEPI         \\ 
	\hline
	Německo & \href{http://www.dpma.de/}{German Patent and Trade Mark Office}  & DPMA         \\ 
	\hline
	Nizozemsko & \href{http://www.rvo.nl/octrooien}{Netherlands Patent Office}  & -         \\ 
	\hline
	Norsko & \href{https://www.patentstyret.no/en/}{Norwegian Industrial Property Office}  & NIPO         \\ 
	\hline
	Nový Zéland & \href{http://www.iponz.govt.nz/}{Intellectual Property Office of New Zealand}  & IPONZ         \\ 
	\hline
	Peru & \href{http://www.indecopi.gob.pe/}{National Institute for the Defense of Competition and Protection of Intellectual Property}  & INDECOPI         \\ 
	\hline
	Polsko & \href{https://uprp.gov.pl/pl}{Urząd Patentowy Rzeczypospolitej Polskiej}  & UPRP         \\ 
	\hline
	Portugalsko & \href{https://inpi.justica.gov.pt/}{Portuguese Institute of Industrial Property}  & -         \\ 
	\hline
	Rakousko & \href{http://www.patentamt.at/}{Austrian Patent Office}  & -         \\ 
	\hline
	Rumunsko & \href{http://www.osim.ro/}{State Office for Inventions and Trademarks}  & OSIM         \\ 
	\hline
	Rusko & \href{https://rospatent.gov.ru/}{Federal Service for Intellectual Property}  & Rospatent         \\ 
	\hline
	Řecko & \href{http://www.obi.gr/el/}{Hellenic Industrial Property Organization}  & HIPO         \\ 
	\hline
	Singapur & \href{http://www.ipos.gov.sg/}{Intellectual Property Office of Singapore}  & IPOS         \\ 
	\hline
	Slovensko & \href{https://www.indprop.gov.sk/}{Industrial Property Office of the Slovak Republic}  & -         \\ 
	\hline
	Slovinsko & \href{http://www.uil-sipo.si/}{Slovenian Intellectual Property Office}  & SIPO         \\ 
	\hline
	Srbsko & \href{http://www.zis.gov.rs/}{Intellectual Property Office of the Republic of Serbia}  & -         \\ 
	\hline
	Španělsko & \href{http://www.oepm.es/}{Spanish Patent and Trademark Office}  & OEPM         \\ 
	\hline
	Švédsko & \href{http://www.prv.se/}{Swedish Intellectual Property Office}  & PRV         \\ 
	\hline
	Švýcarsko & \href{https://www.ige.ch/}{Swiss Federal Institute of Intellectual Property}  & -         \\ 
	\hline
	Turecko & \href{http://www.turkpatent.gov.tr/}{Turkish Patent and Trademark Office}  & Turkpatent         \\ 
	\hline
	Ukrajina & \href{https://ukrpatent.org/en}{Ukrainian Intellectual Property Institute}  & Ukrpatent         \\ 
	\hline
	\end{tabular}
	\caption{Národní patentové úřady a jejich zkratky, část druhá}
	\label{tab:table_offices2}
	\end{table}

\newpage

\subsection{Atributy}\label{subsec:atributy}
\subsubsection{Povinné atributy}

	\begin{table}[H]
	\centering
	\begin{tabular}{|>{\centering\arraybackslash}p{2.2cm}|>{\centering\arraybackslash}p{2cm}|>{\centering\arraybackslash}p{3cm}|>{\centering\arraybackslash}p{2cm}|>{\centering\arraybackslash}p{2.5cm}|} 
	\hline
	\textbf{Země}    & \textbf{Název patentu} & \textbf{Rok přihlášky / publikace} & \textbf{Autor} & \textbf{ID patentu}                \\ 
	\hline
	Kanada & x & x & x & x \\
	\hline
	Česko & x & x & - & x \\
	\hline
	Litva & x & x & x & x \\
	\hline
	Portugalsko & x & x & x & x \\
	\hline
	Španělsko & x & x & x & x \\
	\hline
	Švédsko & - & x & - & x \\
	\hline
	Izrael & x & x & x & x \\
	\hline
	Itálie & x & x & x & x \\
	\hline
	Mexiko & x & x & x & x \\
	\hline
	Polsko & x & x & - & - \\
	\hline
	Anglie & x & x & x & x \\
	\hline
	Rusko & x & x & x & x \\
	\hline
	Peru & x & x & x & x \\
	\hline
	Francie & x & x & x & x \\
	\hline
	\end{tabular}
	\caption{Povinné atributy nacházející se v dostupných patentech}
	\label{tab:table_attributes_critical}
	\end{table}

\subsubsection{Nepovinné atributy}

	\begin{table}[H]
	\centering
	\begin{tabular}{|c|c|c|c|c|} 
	\hline
	\textbf{Země}    & \textbf{Abstrakt} & \textbf{Slovník} & \textbf{Reference} & \textbf{Žadatel} \\
	\hline
	Kanada & - & - & - & x \\
	\hline
	Česko & x & - & x & - \\
	\hline
	Litva & x & - & - & x \\
	\hline
	Portugalsko & x & - & - & x \\
	\hline
	Španělsko & x & - & - & x \\
	\hline
	Švédsko & x & - & - & - \\
	\hline
	Izrael & - & - & - & x \\
	\hline
	Itálie & - & - & - & x \\
	\hline
	Mexiko & x & - & - & x \\
	\hline
	Polsko & x & x & - & - \\
	\hline
	Anglie & - & - & - & - \\
	\hline
	Rusko & - & - & - & - \\
	\hline
	Peru & - & - & - & - \\
	\hline
	Francie & x & - & x & x \\
	\hline
	\end{tabular}
	\caption{Nepovinné atributy nacházející se v dostupných patentech, část první}
	\label{tab:table_attributes_notcrit1}
	\end{table}

	\begin{table}[H]
	\centering
	\begin{tabular}{|c|c|c|c|c|} 
	\hline
	\textbf{Země}    &  \textbf{Adresa} & \textbf{Rodina patentů} & \textbf{Obor} & \textbf{Fulltext} \\
	\hline
	Kanada & x & - & x & - \\
	\hline
	Česko & - & - & x & - \\
	\hline
	Litva & - & - & x & - \\
	\hline
	Portugalsko & - & - & x & - \\
	\hline
	Španělsko & x & - & x & x \\
	\hline
	Švédsko & - & - & x & x \\
	\hline
	Izrael & x & - & - & - \\
	\hline
	Itálie & - & - & - & - \\
	\hline
	Mexiko & - & - & x & - \\
	\hline
	Polsko & - & - & - & - \\
	\hline
	Anglie & - & - & x & - \\
	\hline
	Rusko & - & - & - & - \\
	\hline
	Peru & - & - & x & - \\
	\hline
	Francie & - & - & x & - \\
	\hline
	\end{tabular}
	\caption{Nepovinné atributy nacházející se v dostupných patentech, část druhá}
	\label{tab:table_attributes_notcrit2}
	\end{table}


\subsection{Závěr průzkumu}
Rovnou udělat kapitolu, ve který se aplikují všechny podmínky + se sepíše souhrn počtu patentů, z jakých zemí atp.


\section{Výběr databáze}
\todo{todo}
\subsection{Výběr typu databáze}
\subsection{Výběr z existujících řešení}
\subsection{Závěr průzkumu}