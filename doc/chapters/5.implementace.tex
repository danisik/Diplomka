%CHAPTER
\chapter{Implementace úložiště}
Výsledné řešení by mělo obsahovat dvě databáze, pomocí kterých bude možno vyhledávat patenty, ať už podle určitých atributů (ID, země, obor, a jiné), nebo pomocí full-textu. Vybrané databáze jsou uzpůsobeny pro ukládání velkého množství patentových dat, které byly staženy z národních zdrojů. Pro databázi \textbf{MySQL} bylo navrženo a vytvořeno databázové schéma tak, aby bylo možno použít co nejvíce atributů při filtrování patentů pro statistiky. V \text{MongoDB} bude vytvořena databáze s jednou kolekcí, která bude obsahovat všechny vyfiltrované patenty.
\newline
\indent Import dat bude řešen pomocí vlastní aplikace, která bude filtrovat nevalidní patenty (neexistující povinné atributy, nevalidní \gls{XML} struktura, ...) a importovat validní patenty do databází. Pro MySQL databázi bude potřeba z patentu získat pouze specifické atributy.
\newline
\indent Výsledné řešení bude potřeba jednoduše nasadit na produkční server / počítač, aniž by bylo potřeba instalovat vícero aplikací. Řešení se postará o automatickou instalaci obou databází, jejich inicializaci (vytvoření tabulek, kolekcí, pohledů, ...) a zajistí propojení mezi databází \textbf{MongoDB} a full-textovým vyhledávačem \textbf{Elasticsearch}. S databázemi se zároveň nainstalují i aplikace pro jejich správu.

\section{Implementace databáze}
\todo{todo}

\subsection{MongoDB}
% databázi, kolekci, replica set, verze mongo ?
\todo{todo}

\subsection{MySQL} \label{subsec:mysql_impl}
% verze mysql ?
Pro MySQL byla použita aktuálně nejnovější verze Community serveru (verze 8.0.29). Programový systém \textbf{phpMyAdmin} (verze 5.1.3)  bude použit jako nástroj pro spravování MySQL databáze, ke kterému lze přistoupit pomocí webového prohlížeče. Lze použít i nástroj \textbf{MySQL Workbench} pro správu MySQL databáze, ale pro naše účely bohatě postačí \textbf{phpMyAdmin}.
\newline
\indent Schéma splňuje pouze druhou normální formu. Je to z toho důvodu, že tabulka \textbf{classification} obsahuje některé duplicitní hodnoty ve sloupcích. Duplicitu samozřejmě lze odstranit, ale zvýšilo by to složitost celého systému - \gls{SQL} dotazy by byly složitější a pomalejší v případě příkazu \textit{SELECT}, který by vyhledával data z několika tabulek, spojených příkazem \textit{JOIN}.
\newline

\noindent Databázové schéma pro MySQL lze vidět na obrázku č. \ref{fig:mysql_schema}.
\begin{figure}[H]
\centering
\includegraphics[width=12cm]{img/eer}
\caption{Schéma pro MySQL databázi.}
\label{fig:mysql_schema}
\end{figure}

\subsubsection{Tabulka patents}
Tabulka \textbf{patents} uchovává všechny národní patenty, které byly poskytnuty zdarma a obsahují všechny povinné atributy.

\subsubsection{Tabulka classification}
Tabulka \textbf{classification} slouží k uchovávání \gls{IPC} klasifikace daného patentu, konkrétně jeho sekci, třídu a podtřídu (skupina a podskupina vybrána nebyla, protože se vyskytovala v patentech jen ojediněle). 

\subsubsection{Tabulka patent\_classification}
Tabulka \textbf{patent\_classification} má kardinalitu \textit{M:N} a slouží jako propojení tabulky \textbf{patents} a tabulky \textbf{classification}. Jeden patent může mít více klasifikací (například se mohlo změnit označení v průběhu let, nebo patent pokrývá více oborů) a jedna klasifikace může být u více patentů.

\subsubsection{Tabulka inventors}
Tabulka \textbf{inventors} slouží k uchovávání jmen autorů patentů. Kardinalita mezi tabulkou \textbf{patents} a tabulkou \textbf{inventors} je \textit{1:N}, protože pro jeden patent může existovat více autorů.

\subsubsection{Tabulka applicants}
Tabulka \textbf{applicants} slouží k uchovávání jmen žadatelů patentů. Ačkoli existuje tabulka pro autory, tak může existovat scénář, ve kterém bude potřeba vyhledávat žadatele pro daný patent. Kardinalita mezi tabulkou \textbf{patents} a tabulkou \textbf{applicants} je \textit{1:N}, protože pro jeden patent může existovat více žadatelů.

\subsubsection{Pohledy}
Pohled je databázový objekt, který uživateli poskytuje data ve stejné podobě jako tabulka. Stručněji řečeno, je to struktura uchovávající \gls{SQL} dotaz, který se většinou dotazuje dané tabulky na specifická data.
\newline
\indent V databázi pro patenty bylo vytvořeno celkem devět pohledů, kdy každý z nich reprezentuje jeden scénář, který je použit při ověřování efektivního vytěžování (viz kapitola č. \ref{sec:efektivni_vytezovani_sql}).

\section{Import dat}
% nezmiňovat jak vypadají projekty, jen jak se to přiblžině dělalo a co všechno se muselo dělat. zmínit že tam byly i obrázky a další kokotiny ?
% popsat jak se importovala data (případně i postup importu, že existuje tenhle projekt pomocí kterého se to importovalo)
% vypsat počet hodnot v tabulkách
\todo{todo}

\section{Nasazení úložiště}
Celý modul našeho úložiště čítá celkem 5 nástrojů - MySQL, MongoDB, Elasticsearch, phpMyAdmin a Mongo-express. Zároveň bude potřeba nastavit připojení mezi MongoDB a Elasticsearch tak, aby při přidání nového patentu do databáze byl následně zaindexován ve full-textovém vyhledávači. Pokud bychom měli po každém uživateli chtít instalaci všech těchto nástrojů a k tomu stahovat další nástroje pro vytvoření připojení mezi MongoDB a Elasticsearch, tak to zabere mnoho času a existuje zde velká pravděpodobnost že některé nástroje nebudou kompatibilní s aktuální verzí operačního systému. Z těchto důvodů byl zvolen software \textbf{Docker}, pomocí kterého se provede veškerá instalace a nastavení automaticky.

\subsection{Docker}
\textbf{Docker} je jeden z nejznámějších open-source nástrojů pro dodání aplikací v balíčkách zvaných \textit{kontejner}. \textbf{Docker} využívá virtualizaci na úrovni operačního systému, čímž je výrazně snížena režie na rozdíl od klasických virtuálních strojů. Existují i jiné alternativy než docker, ale díky velké popularitě a fanouškovské základně byl vybrán právě docker.
\newline
\indent Definice a instalace aplikací je zajištěna pomocí nástroje \textbf{Docker compose}. Definice probíhá pomocí souboru s názvem \textit{docker-compose.yml}, který využívá \gls{YAML} formát pro serializaci strukturovaných dat. V docker-compose.aml souboru pro tento modul je definováno celkem devět aplikací a dva nastavovací moduly.
\todo{obrázek dockeru - po Docker- MongoDB a Elastic....}

\subsubsection{Docker - Elasticsearch}
Pro Elasticsearch byly nadefinovány dvě aplikace:
\begin{itemize}
\item \textbf{elasticsearch} - Oficiální image full-textového vyhledávače Elasticsearch ve verzi 8.2.0. V nastavení byla vypnuto zabezpečení kvůli problémům se sadou replik v MongoDB.
\item \textbf{elasticvue} - Grafické uživatelské rozhraní pro Elasticsearch ve verzi 0.39.0, dostupné na localhostu na portu 8080.
\end{itemize}

\subsubsection{Docker - MongoDB}
Pro MongoDB byly nadefinovány dvě aplikace a jeden nastavovací modul:
\begin{itemize}
\item \textbf{mongo} - Oficiální image databáze MongoDB ve verzi 5.0.6. V databázi bylo nutné nastavit sadu replik, aby bylo možné vytvořit propojení mezi MongoDB a Elasticsearch.
\item \textbf{mongo-express} - Grafické uživatelské rozhraní pro MongoDB ve verzi 0.54.0, dostupné na localhostu na portu 8081.
\item \textbf{mongo-setup} - Nastavovací modul pro MongoDB, který po inicializaci databáze inicializuje sadu replik a následně vytvoří novou databázi a kolekci pro patenty.
\end{itemize}

\subsubsection{Docker - MySQL}
Pro MySQL byly nadefinovány dvě aplikace:
\begin{itemize}
\item \textbf{mysql} - Oficiální image databáze MySQL ve verzi 8.0.29. Při inicializaci databáze se vytvoří všechny tabulky i pohledy.
\item \textbf{phpmyadmin} - Grafické uživatelské rozhraní pro MySQL ve verzi 5.1.3, dostupné na localhostu na portu 8082. V nastavení byl navýšen limit pro import souborů na 600 MB z původních 2 MB.
\end{itemize}

\subsubsection{Docker - MongoDB a Elasticsearch konektor}
Pro konektor mezi MongoDB a Elasticsearch byly nadefinovány tři aplikace a jeden nastavovací modul:
\begin{itemize}
\item \textbf{broker} - Komunitní verze Apache Kafka od firmy Confluent ve verzi 6.1.0. Apache Kafka je distribuované úložiště událostí a platforma pro zpracování datových proudů (streamů).
\item \textbf{zookeeper} - Nástroj pro Apache Kafka ve verzi 6.1.0, který funguje jako centralizovaná služba a slouží k údržbě jmenných a konfiguračních dat. Zároveň zajišťuje i flexibilní a robustní synchronizaci v rámci distribuovaných systémů.
\item \textbf{connect} - Nástroj pro Apache Kafka ve verzi 6.1.0, který má za úkol propojit externí systémy s Apache Kafka za pomoci konektorů.
\item \textbf{connect-setup} - Nastavovací modul pro Kafka connect, který po inicializaci Kafka connect vytvoří dva konektory:
	\begin{itemize}
	\item \textit{mongo-source-connector} - Konektor, který aktivně získává nová data z MongoDB a vkládá je do Kafka brokera.
	\item \textit{elasticsearch-sink-connector} - Konektor, který aktivně získává nová data z Kafka brokera a vkládá je do Elasticsearch.
	\end{itemize} 
\end{itemize}

\subsection{Inicializace MySQL}
\todo{todo}
% popsat proč existují jen výsledné soubory s daty, ne raw data (že pro anglii je tolik a tolik souborů, který obsahují víc jak jeden patent, blabla).

\subsection{Inicializace MongoDB}
\todo{todo}
% popsat proč existují jen výsledné soubory s daty, ne raw data.

\subsection{Propojení MongoDB a Elasticsearch}
\todo{todo}
% zmínit mongo-connector a kafku, s jakou verzí to fungovalo atp., že jedno řeší mapping, druhý už ne