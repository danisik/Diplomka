%%%%%%%%%%%%%%%%%%%%%%%%%%%%%%%%%%%%%%%%%%%%%%%%%%%%%%%%%%
%
% Vzor pro sazbu kvalifikační práce
%
% Západočeská univerzita v Plzni
% Fakulta aplikovaných věd
% Katedra informatiky a výpočetní techniky
%
% Petr Lobaz, lobaz@kiv.zcu.cz, 2016/03/14
%
%%%%%%%%%%%%%%%%%%%%%%%%%%%%%%%%%%%%%%%%%%%%%%%%%%%%%%%%%%

% Možné jazyky práce: czech, english
% Možné typy práce: BP (bakalářská), DP (diplomová)
\documentclass[czech,DP]{thesiskiv}

% Definujte údaje pro vstupní strany
%
% Jméno a příjmení; kvůli textu prohlášení určete, 
% zda jde o mužské, nebo ženské jméno.
\author{Bc. Vojtěch Danišík}
\declarationmale

% Název práce
\title{Tvorba rozsáhlých úložišť patentových dat}

% 
% Texty abstraktů (anglicky, česky)
%
\abstracttexten{Creation of large-scale patent data repositories. The aim of the diploma thesis is to get acquainted with the available sources of patent data and to create extensive local repositories of patent data enabling their effective searching and mining. The first part of the thesis thoroughly describes the types of patents, existing data sources and file formats in which patents are stored. Subsequently, the applicable technologies for searching and mining are described. The second part of the thesis is devoted to the selection of usable data and the implementation of selected technologies. Several queries and scenarios have been created to test efficient mining. The results of the testing are part of this work.
}

\abstracttextcz{Cílem diplomové práce je seznámit se s dostupnými zdroji dat o patentech a vytvořit rozsáhlá lokální úložiště patentových dat umožňující jejich efektivní prohledávání a vytěžování. První část práce důkladně popisuje typy patentů, existující zdroje dat a formáty souborů, ve kterých se patenty ukládají. Následně jsou popsány použitelné technologie pro prohledávání a vytěžování. Druhá část práce se věnuje výběru použitelných dat a implementaci vybraných technologií. Pro otestování efektivního vytěžování  bylo vytvořeno několik dotazů a scénářů. Výsledky testování jsou součástí této práce.
}

% Na titulní stranu a do textu prohlášení se automaticky vkládá 
% aktuální rok, resp. datum. Můžete je změnit:
%\titlepageyear{2016}
%\declarationdate{1. března 2016}

% Ve zvláštních případech je možné ovlivnit i ostatní texty:
%
%\university{Západočeská univerzita v Plzni}
%\faculty{Fakulta aplikovaných věd}
%\department{Katedra informatiky a výpočetní techniky}
%\subject{Bakalářská práce}
%\titlepagetown{Plzeň}
%\declarationtown{Plzni}

%%%%%%%%%%%%%%%%%%%%%%%%%%%%%%%%%%%%%%%%%%%%%%%%%%%%%%%%%%
%
% DODATEČNÉ BALÍČKY PRO SAZBU
% Jejich užívání či neužívání záleží na libovůli autora 
% práce
%
%%%%%%%%%%%%%%%%%%%%%%%%%%%%%%%%%%%%%%%%%%%%%%%%%%%%%%%%%%

% Zařadit literaturu do obsahu
\usepackage[nottoc,notlot,notlof]{tocbibind}

% Umožňuje vkládání obrázků
\usepackage[pdftex]{graphicx}

% Odkazy v PDF jsou aktivní; navíc se automaticky vkládá
% balíček 'url', který umožňuje např. dělení slov
% uvnitř URL
\usepackage[pdftex]{hyperref}
\hypersetup{colorlinks=true,
  unicode=true,
  linkcolor=black,
  citecolor=black,
  urlcolor=black,
  bookmarksopen=true}

% Při používání citačního stylu csplainnatkiv
% (odvozen z csplainnat, http://repo.or.cz/w/csplainnat.git)
% lze snadno modifikovat vzhled citací v textu
\usepackage[numbers,sort&compress]{natbib}
\usepackage[dvipsnames]{xcolor}
\usepackage{array}
\usepackage{float}
\usepackage{textcomp}
\usepackage[colorinlistoftodos,prependcaption,textsize=tiny]{todonotes}
\usepackage{listings,xcolor}
\usepackage{verbatim}
\usepackage{fancyhdr}
\usepackage{forest}

% Acronyms
\usepackage[nomain, abbreviations, automake]{glossaries-extra}
\makeglossaries
\setabbreviationstyle{long-short}
\setabbreviationstyle[acronym]{short}
% Databáze
\newacronym{DBMS}{DBMS}{Database Management Systems}
\newacronym{ACID}{ACID}{Atomicity, Consistency, Isolation, Durability}
\newacronym{JSON}{JSON}{JavaScript Object Notation}
\newacronym{SQL}{SQL}{Structured Query Language}
\newacronym{XML}{XML}{Extensible Markup Language}
\newacronym{BSON}{BSON}{Binary JSON}
\newacronym{GPL}{GPL}{GNU General Public License}
\newacronym{KQL}{KQL}{Kibana Query Language}
\newacronym{ASCII}{ASCII}{American Standard Code for Information Interchange}
\newacronym{CRUD}{CRUD}{Create, Read, Update, Delete}
\newacronym{MQL}{MQL}{MongoDB Query Language}
\newacronym{API}{API}{Application Programming Interface}

% Návrh uložiště
\newacronym{EPO}{EPO}{European Patent Office}
\newacronym{WIPO}{WIPO}{World Intellectual Property Organization}
\newacronym{USPTO}{USPTO}{United States Patent and Trademark Office}
\newacronym{IPC}{IPC}{International Patent Classification}

% Implementace
\newacronym{YAML}{YAML}{YAML Ain't Markup Language}
\newacronym{HTTP}{HTTP}{Hypertext Transfer Protocol}
\newacronym{PDF}{PDF}{Portable Document Format}
\newacronym{FTP}{FTP}{File Transfer Protocol}

% Rozšiřitelnost
\newacronym{CSV}{CSV}{Comma-separated values}
\newacronym{URL}{URL}{Uniform Resource Locator}

\lstdefinestyle{sqlstyle}{
language = sql,
frame = single,
basicstyle = \small\ttfamily,
breaklines = true,
showtabs = false,
showstringspaces = false,
stringstyle = \color{red},
identifierstyle = \color{blue},
commentstyle = \color{gray},
numberstyle = \color{yellow},
emph =[1]{inventors, ., patents, applicants, classification},
emphstyle =[1]\color{black}\bfseries,
}

%%%%%%%%%%%%%%%%%%%%%%%%%%%%%%%%%%%%%%%%%%%%%%%%%%%%%%%%%%
%
% VLASTNÍ TEXT PRÁCE
%
%%%%%%%%%%%%%%%%%%%%%%%%%%%%%%%%%%%%%%%%%%%%%%%%%%%%%%%%%%
\begin{document}
%
\maketitle
\pagestyle{empty}
\tableofcontents
\pagestyle{plain}
\addtocontents{toc}{
\protect\thispagestyle{empty}} 

\thispagestyle{empty}
\setcounter{page}{0} 

%CHAPTER
\chapter{Úvod}
\todo{todo}
zmínit deployment - docker atp, aby se to mohlo narvat do teorie
%CHAPTER
\chapter{Patent}
\todo{todo}
základní informace + uložiště dat + info o zdrojích (patentové úřady atp) \newline
\section{Patent vs Užitný vzor}
\section{Patent vs Průmyslový vzor}
\section{Patent vs Ochranná známka}
\section{Patent vs Autorská práva}


%%%%%%%%%%%%%%%%%%%%%%%%%%%%%%%%%%%%%%%%%%%%%%%%%%%%%%%%%%%%

%CHAPTER
\chapter{Databáze}

Termín databáze označuje organizovanou kolekci strukturovaných informací nebo dat, která jsou typicky ukládána elektronicky v počítačovém systému. Data / informace lze nazvat jako fakta vztahující se k libovolnému uvažovanému objektu. Typický příklad objektu je člověk, jehož fakta jsou: jméno, věk, výška, váha a mnoho dalších \cite{guru99Database}.\newline

\section{Systém řízení báze dat}
Pro správu dat v databázi a její řízení je potřeba komplexní software, který se nazývá \textbf{Systém Řízení Báze Dat} (SŘBD, anglicky \gls{DBMS}). SŘBD slouží jako interface mezi samotnou databází a koncovým uživatelem (může být i program), umožňující jak vytěžování a aktualizaci dat, tak i možnosti nastavení záloh a jiných administrativních operací \cite{OracleDB}. V dnešním světe existuje několik různých \gls{DBMS} (například Relační \gls{DBMS}, Objektově-orientované \gls{DBMS}).
\begin{figure}[h!]
\centering
\includegraphics[width=10cm]{img/databaze/dbms}
\caption{Systém řízení báze dat}
\label{fig:dbms}
\end{figure}

\section{Komponenty databáze}
Všechny databáze sestávají z pěti základních komponent, nehledě na použitý typ databáze \cite{TechTargetDB, guru99Database}:
\begin{itemize}
\item \textbf{Hardware} - Fyzické stroje (počítače, servery, pevné disky, ...) na kterých běží databázový software.
\item \textbf{Software} - Databázový software poskytuje uživateli / programu kontrolu nad databází. Zahrnuje to samotný databázový software, operační systém, software pro správu sdílení dat mezi uživately a programy pro přístup k datům v databázi.
\item \textbf{Data} - Nezpracované a neorganizované fakty, které je potřeba zpracovat. Administrátor databáze organizuje tyto data a dává jim význam. Data se obecně skládají hlavně z faktů, observací, percepcí, čísel, znaků a mnoho dalších.
\item \textbf{Jazyk} - Typický příklad použití jazyku je přístup k datům, přidávání nových dat, úpravu již existujících dat z databáze. Uživatel / program napíše specifické příkazy v jazyku pro přístup k datům (Database Access Language) a tyto příkazy následně pošle databázi ke zpracování. Více viz kapitola č. \ref{sec:jazyky}.
\item \textbf{Procedury} - Procedura obsahuje předpřipravený seznam příkazů, které se následně vykonávají po zavolání dané procedury. 
\end{itemize}

\section{Typy databází}
V dnešním světě existuje mnoho různých typů databází. Výběr nejlepšího typu databáze pro konkrétní organizaci závisí na tom, jak organizace zamýšlí data používat. V této kapitole je vypsáno pouze pár typů, protože vznikají stále nové, méně známé typy databází, které jsou tvořeny pro specifické požadavky (například finanční, věděcké) \cite{matillionTypeDB, OracleDB}.
\subsection{Relační databáze}
\todo{Struktura ? Data model ? Výhody nevýhody ?}
Název relační databáze pochází ze způsobu, jakým jsou data uložena, a to ve více souvisejících tabulkách. Data v tabulkách jsou uložena v řádcích a sloupcích. Relační databáze jsou velice spolehlivé a podporují všechny čtyři žádoucí vlastnosti databázových transakcí \gls{ACID}. Pro co nejefektivěnjší využití tohoto typu databáze je potřeba ukládat pouze dobře strukturovaná data, pro částečně strukturovaná či nestrukturovaná data je vhodné použít například grafové nebo dokumentově založené databáze. Typické relační databáze jsou například: Microsoft SQL Server, Oracle Database, MySQL, PostgreSQL. Ukázku relační databáze lze vidět na obrázku č. \ref{fig:db_img_relational}.
	\begin{figure}[H]
	\centering
	\includegraphics[width=14cm]{img/databaze/relational_db}
	\caption{Ukázka relační databáze}
	\label{fig:db_img_relational}
	\end{figure}
\subsection{Objektově-orientovaná databáze}
\todo{Struktura ? Data model ? Výhody nevýhody ?}
Objektově-orientovaná databáze je založena na objektově-orientovaném programování, kdy data a všechny jejich atributy a metody jsou svázány dohromady jako objekt. Stejně jako relační databáze, i objektově-orientované databáze odpovídají standardům \gls{ACID}. Typické příklady jsou například: ObjectStore, ConceptBase. Ukázku objektově-orientované databáze lze vidět na obrázku č. \ref{fig:db_img_oo}.
	\begin{figure}[H]
	\centering
	\includegraphics[width=12cm]{img/databaze/oo_db}
	\caption{Ukázka objektově-orientované databáze}
	\label{fig:db_img_oo}
	\end{figure}
\subsection{NoSQL databáze}
\todo{Struktura ? Data model ? Výhody nevýhody ?}
NoSQL je široká kategorie databází, které nepoužívají \gls{SQL} jako svůj primární jazyk pro přístup k datům. Tyto typy databází jsou také někdy označovány jako nerelační databáze. V NoSQL databázích se pracuje s nestrukturovanými a polostrukturovanými sadami distribuovaných dat. Jednou z výhod je, že vývojáři mohou provádět změny databáze za běhu, aniž by to ovlivnilo aplikace, které databázi používají.
\subsection{Databáze Klíč-Hodnota}
\todo{Struktura ? Data model ? Výhody nevýhody ?}
Databáze klíč-hodnota poskytuje nejjednodušší možný NoSQL datový model. Data jsou uložená jako pár klíč - hodnota ve slovníku / mapě, kdy klíč je indexem. Hodnota může být například celé číslo, řetězec, struktura \gls{JSON} nebo pole. Typické příklady jsou: Redis, Riak, LevelDB. Ukázku databáze klíč-hodnota lze vidět na obrázku č. \ref{fig:db_img_keyvalue}.
	\begin{figure}[H]
	\centering
	\includegraphics[width=10cm]{img/databaze/keyvalue_db}
	\caption{Ukázka databáze klíč-hodnota.}
	\label{fig:db_img_keyvalue}
	\end{figure}
\subsection{Grafová databáze}
\todo{Struktura ? Data model ? Výhody nevýhody ?}
Grafová databáze je typem NoSQL databáze, která je založená na teorii grafů. Data jsou reprezentována jako uzly, hrany zase reprezentují vztahy mezi daty. Graf lze procházet podél určitých typů hran nebo přes celý graf. Procházení spojení nebo relací je velmi rychlé, protože vztahy mezi uzly se nepočítají v době dotazu, ale jsou v databázi trvalé. Typické příklady jsou: Neo4j, OrientDB, Microsoft Azure CosmosDB. Ukázku grafové databáze lze vidět na obrázku č. \ref{fig:db_img_graph}.
	\begin{figure}[H]
	\centering
	\includegraphics[width=8cm]{img/databaze/graph_db}
	\caption{Ukázka grafové databáze}
	\label{fig:db_img_graph}
	\end{figure}
\subsection{Dokumentová databáze}
\todo{Struktura ? Data model ? Výhody nevýhody ?}
Databáze dokumentů jsou typem NoSQL databáze a  jsou navržené pro ukládání, načítání a správu informací orientovaných na dokumenty. Dokumenty jsou obvykle uloženy ve formátu \gls{XML}, \gls{JSON}, \gls{BSON}. 
Typické příklady jsou: MongoDB, Amazon DocumentDB, Elasticsearch. Ukázku dokumentové databáze lze vidět na obrázku č. \ref{fig:db_img_document}.
	\begin{figure}[H]
	\centering
	\includegraphics[width=11cm]{img/databaze/document_db}
	\caption{Ukázka dokumentově orientované databáze}
	\label{fig:db_img_document}
	\end{figure}

\section{Existující řešení}
Pro vybrané typy databáze existují mnoho databázových řešení, které lze zmínit. V této kapitole se budeme zabývat především těmi nejznámějšími pro daný typ databáze, a které jsou zdarma ke stažení a používání. Pro každý typ databáze byly vybrány maximálně dvě řešení.
\subsection{MySQL}
MySQL je multiplatformní databáze uplatňující relační databázový model. Komunikace s databází (získávání dat, vytváření objektů, ...) probíhá pomocí jazyka \gls{SQL}, který je rozšířen o nové funkce. Nejnovější verze MySQL je open-source, což znamená, že kdokoliv může používat a libovolně upravovat MySQL systém, aniž by musel cokoliv platit. V případě změny zdrojových kódů je potřeba nastudovat podmínky užívání definované licencí \gls{GPL} \cite{mysql}.
\newline
\indent Od samých počátků bylo MySQL optimalizováno především na rychlost i za cenu některých zjednodušení (například způsob zálohování dat). Díky tomuto lze provozovat jednoduché servery na počítači společně s jinýma aplikacema, případně jiné databáze. Server lze nakonfigurovat tím způsobem, že může využívat veškerou paměť, procesorový čas i vstupně výstupní kapacity.
\newline

\noindent MySQL server může být využit dvěma způsoby:
\begin{itemize}
\item \textbf{Klient / server} - vícevláknový \gls{SQL} server, který podporuje různé back-endy, několik různých klientských programů a knihoven a mnoho dalšího.
\item \textbf{Věstavěná knihovna} - vícevláknová věstavěná knihovna, kterou lze propojit do své aplikace a získat tím menší, rychlejší a snadněji spravovatelný samostatný produkt.
\end{itemize}
\subsection{PostgreSQL}
Relační \todo{todo}
\subsection{ConceptBase}
OO \todo{todo}
\subsection{LevelDB}
KeyValue \todo{todo}
\subsection{Redis}
KeyValue \todo{todo}
\subsection{MongoDB}
Document \todo{todo}
\subsection{Elasticsearch}
Document \todo{todo}
\subsection{Neo4j}
Graf \todo{todo}

\section{Jazyky} \label{sec:jazyky}
Databázové jazyky, jinak známé jako dotazovací jazyky, jsou klasifikací programovacích jazyků, které se používají k definování a přístupu k databázím. Pomocí těchto jazyků dokáže uživatel získávat nebo spravovat data v databázích. V dnešní době se jazyky (například \gls{SQL}) mohou skládat ze čtyř podjazyků, kdy každý slouží k jinému účelu v rámci vykonávání příkazů \cite{indeedDBLanguage, begginersBookDBLanguage}:
\begin{itemize}
\item \textbf{Data definition language} (DDL) - DDL umí vytvářet jednotlivé komponenty databázového schématu (tabulky, soubory, indexy, ...), které tvoří strukturu reprezentující organizaci dat v databázi. Dostupné příkazy pro jazyk DDL:
	\begin{itemize}
	\item \textbf{CREATE} - Vytvoření nového objektu (tabulka, index, ...).
	\item \textbf{ALTER} - Změna struktury objektu.
	\item \textbf{DROP} - Smazání objektu.
	\item \textbf{RENAME} - Změna názvu objektu.
	\item \textbf{TRUNCATE} - Smazání podobjektů v objektu (například záznamy v tabulce).
	\end{itemize}
\item \textbf{Data manipulation language} (DML) - DML slouží pro manipulaci s daty, které se nachází v již existující databázi. Dostupné příkazy pro jazyk DML:
	\begin{itemize}
	\item \textbf{SELECT} - Získání záznamů (dat) z tabulky.
	\item \textbf{INSERT} - Vložení nového záznamu (dat) do tabulky.
	\item \textbf{UPDATE} - Úprava existujícího záznamu v tabulce.
	\item \textbf{DELETE} - Smazání záznamu z tabulky.
	\end{itemize}
\item \textbf{Data control language} (DCL) - Pomocí DCL lze kontrolovat přístupy a práva k datům, které jsou uloženy v databázi. Uživateli lze nastavit práva k jednotlivým DML příkazům nad tabulkama / procedurama (například uživatel bude mít přístup pouze k příkazu SELECT nad tabulkou "TABULKA"). Dostupné příkazy pro jazyk DCL:
	\begin{itemize}
	\item \textbf{GRANT} - Přidání práv uživateli nad danou tabulkou / procedurou. 
	\item \textbf{REVOKE} - Odebrání práv uživateli nad danou tabulkou / procedurou.
	\end{itemize}

\item \textbf{Transaction control language} (TCL) - TCL spravuje transakce v databázi. Transakce obsahuje jeden či více DML příkazů nad tabulkama, které se vykonávají po sobě. Všechny příkazy musí být úspěšně provedeny, aby bylo možné transakci označit za úspěšnou. Ukázka jedné transakce viz obrázek č. \ref{fig:tcl_savepoint}. Dostupné příkazy pro jazyk TCL:
	\begin{itemize}
	\item \textbf{COMMIT} - Potvrzení transakce, změny provedené v transakci jsou permanentní a nejdou vzít zpět.
	\item \textbf{ROLLBACK} - Vezme zpět veškerou práci v aktuální transakci. Lze se vrátit na začátek transakce nebo k SAVEPOINTu.
	\item \textbf{SAVEPOINT} - Nastavení bodu v transakci, ke kterému se lze v budoucnu vrátit pomocí ROLLBACK.
	\end{itemize}
\end{itemize}
	\begin{figure}[H]
	\centering
	\includegraphics[width=14cm]{img/databaze/tcl_savepoint}
	\caption{Ukázka jedné transakce (bez commitu)}
	\label{fig:tcl_savepoint}
	\end{figure}

\noindent Níže v kapitolách jsou popsány příklady dnešních jazyků.
\subsection{Structured Query Language}
Structured Query Language (\gls{SQL}) je jazyk pro komunikaci s databázema, v dnešní době standard pro relační databázové systémy. Pomocí \gls{SQL} příkazů lze například vytvářet nové objekty v databázi, upravovat existující data v tabulkách nebo vytvářet různá integritní omezení a triggery. Většina existujících databázových systémů používá upravený \gls{SQL} jazyk, který navíc obsahuje dodatečná rozšíření pro splnění požadavků v jejich systémech.

\subsubsection{Syntax}
Syntaxe \gls{SQL} se skládá z unikátního seznamu pravidel a směrnic. Při psaní příkazů zde nehraje roli citlivost písma (příkazy select a SELECT jsou záměnné). Dotazy lze psát na jednu nebo více řádek, které musí / můžou být zakončené středníkem (záleží na pravidlech používaného systému). Na obrázku č. \ref{fig:sql} lze vidět příklad dotazu, který získá jména a příjmení uživatelů z tabulky 'user' s datem narození po roce 2000.
	\begin{figure}[H]
	\centering
	\includegraphics[width=13cm]{img/databaze/sql}
	\caption{Příklad \gls{SQL} dotazu.}
	\label{fig:sql}
	\end{figure}

\noindent Dotazy lze zanořovat do sebe, kdy výsledek jednoho dotazu jde použit jako podmínka pro druhý dotaz, viz obrázek č. \ref{fig:sql2}.
	\begin{figure}[H]
	\centering
	\includegraphics[width=14cm]{img/databaze/sql2}
	\caption{Příklad zanořeného \gls{SQL} dotazu.}
	\label{fig:sql2}
	\end{figure}

\subsection{MongoDB Query Language}
MongoDB Query Language (\gls{MQL}) je jazyk pro získávání dat z MongoDB dokumentových databází. Dotazy zde poskytují jednoduchost v procesu načítání dat z databáze, stejně jako tomu je u \gls{SQL}. Při provádění dotazů lze také použít kritéria nebo podmínky, kterými lze načíst konkrétní data z databáze. Jazyk také podporuje \gls{CRUD} operace. Výsledky můžeme třídit, seskupovat, fitrovat a spočítat jejich četnost za pomoci agregační pipeline (zřetězeného zpracování). \gls{MQL} podporuje transakce více dokumentů \cite{mongo_lang}.

\subsubsection{Syntax}
Syntaxe \gls{MQL} je intuitivní a jednoduchá na používání i pro velice složité dotazy, protože ta samá syntaxe se používá i pro uložené dokumenty v databázi. Příklad syntaxe pro vytváření, čtení, úpravu a mazání dokumentů (\gls{CRUD}) lze vidět na obrázku č. \ref{fig:crud}.
	\begin{figure}[H]
	\centering
	\includegraphics[width=12cm]{img/databaze/crud}
	\caption{Příklady \gls{CRUD} operací v MongoDB.}
	\label{fig:crud}
	\end{figure}

\subsection{Cypher Query Language}
Cypher je dotazovací jazyk pro grafovou databázi Neo4j a umožňuje získávat data z grafů. Tento jazyk byl inspirován hlavně \gls{SQL} - uživatel se zaměřuje pouze na to, jaká data chce získat, ne jak je má získat. Cypher je unikátní v tom, že poskytuje vizuální způsob, jak sladit vzory a vztahy \cite{cypher}.

\subsubsection{Syntax}
Cypher využívá \gls{ASCII}-art typ syntaxe, což je umění, které pracuje s počítačovým textem jako s výtvarným médiem (například obrázky se skládají ze znaků kódu \gls{ASCII}). Syntaxi lze vidět na obrázku č. \ref{fig:cypher_syntax}. Pro jednotlivé uzly se používají kruhové závorky, pro vztah se používá šipka s hranatýma závorkama obsahující vztah prvního uzlu s druhým uzlem.
	\begin{figure}[H]
	\centering
	\includegraphics[width=14cm]{img/databaze/cypher_syntax1}
	\caption{Syntaxe jazyka Cypher.}
	\label{fig:cypher_syntax}
	\end{figure}
\noindent Na obrázku č. \ref{fig:cypher_sample} lze vidět jednoduchý dotaz, který hledá výsledný uzel pro vstupní uzel, kterým je člověk se jménem 'Dan', a vztahu 'LOVES' mezi uzly.
	\begin{figure}[H]
	\centering
	\includegraphics[width=14cm]{img/databaze/sample-cypher}
	\caption{Dotaz v jazyce Cypher.}
	\label{fig:cypher_sample}
	\end{figure}
%%%%%%%%%%%%%%%%%%%%%%%%%%%%%%%%%%%%%%%%%%%%%%%%%%%%%%%%%%%%

%\chapter{Docker}
%\todo{todo}
%Popsat docker jako takový - kontejnery, image, docker compose, ...\newline
%\section{Kontejner vs Virtuální stroj}
\chapter{Návrh úložiště}

%This section provides a general overview of the state of today’s bibliographic
%databases and describes different types of bibliographical databases. First,
%we take a look at publication databases and show some examples of how they
%compare to each other. Next, we describe databases storing bibliographic
%information about patent data and provide some concrete examples of these
%data sources.

%Zdroje (které ano, které ne + proč) + udělat stejnou tabulku jak v excelu + zmínit stránku ze který jsem čerpal informace (wipo.int)\todo{todo}\newline
%Nezapomenout zmínit meze let, ve kterých se jednotlivé patenty z daných zemí nachází \todo{todo}

\indent Hlavní motivací pro vytvoření této práce je vytěžování patentů pro účely zjišťování existence napříkad různých technologických vynálezů či algoritmů. Pomocí těchto informací lze zjistit, zda například má smysl vymýšlet nový algoritmus pro určitý problém a neexistuje k němu jiné, lepší řešení, případně vymyslet modifikaci, která zajistí lepší výsledky.
\newline

\noindent Dále je potřebat definovat, co vlastně znamená pojem efektivní vytěžování. Vytěžování lze označit za efektivní, pokud budou splněny tyto podmínky:
\begin{itemize}
\item \textbf{Rychlost} - Vyhledávání musí probíhat v rámci jednotek až desítek sekund (případně jednotky minut, záleží na celkovém počtu patentů a na hardwarové konfiguraci serveru).
\item \textbf{Stabilita} - Server musí být stabilní a nesmí padat při práci s velkým množstvím dat, zvlášť při vyhledávání s použitím složitých dotazů (nařípkad hledání přes více tabulek).
\end{itemize}

\noindent V následujících kapitolách bude popsán postup výběru zdrojů patentů a patentových dat. Následně budou vybrány typy databáze, které budou vhodné pro uložení vybraných patentových dat a následné zvolení existujících řešení.

\section{Výběr patentů}
Při výběru patentů byly stanoveny tři podmínky, které museli být splněny:
\begin{itemize}
\item \textbf{Dostupnost} - Patenty musí být dostupné z online stránek / databází bez poplatků.
\item \textbf{Datum} - Patentová přihláška nebo publikace patentu musí být podána alespoň v roce 2000, všechny ostatní patenty budou vyfiltrovány.
\item \textbf{Atributy} - Všechny patenty musí obsahovat povinné atributy (viz kapitola č. \ref{subsec:atributy}).
\end{itemize}

\subsection{Zdroje dat}
V dnešním světě existuje několik desítek až stovek patentových zdrojů dat, od webových vyhledávačů v databázi až po plný export databáze s patenty. Velké organizace, jako například \gls{EPO}, \gls{WIPO}, \gls{USPTO}, udržují jedny z největších patentových databází (desítky až stovky milionů patentů), ve kterých lze vyhledávat velké množství informací zdarma za použití webových vyhledávačů na dané stránce organizace. Lze zde najít všechny typy patentů (přihlášky, publikace), národní patenty i patenty registrované například u \gls{EPO}. V případě exportu databází, \gls{USPTO} poskytuje plný export svých databází veřejnosti pro libovolné používání, zcela zdarma. Využití těchto zdrojů dat by bylo určitě skvělé, ale tyto zdroje byly nedávno použity a rozebrány v jiné diplomové práci, proto je vhodné se spíše zaměřit na národní zdroje dat patentů.
\newline
\indent Národní databáze patentů dané země obsahuje všechny národní patenty, některé dokonce i patenty z jiných zemí registrovaných u \gls{EPO}. 
\newline
\indent Při průzkumu bylo zkoumáno celkem 51 národních zdrojů dat (patentových úřadů). V tabulkách č. \ref{tab:table_offices1} a \ref{tab:table_offices2} lze vidět název země, název patentového úřadu v dané zemi, zkratku patentového úřadu (pokud nějakou má) a jestli patentový úřad poskytoval data nebo ne. U každého patentového úřadu byl procházen její oficiální web a zkoumán na dostupnost patentových dat. Většina úřadů má na svých stránkách vyhledávač pro procházení vlastní databáze patentů, ale jen zlomek z nich poskytoval použitelná data zadarmo. Tyto data byla většinou schována pod neodpovídajícím názvem článku / příspěvku, a některé dokonce poskytovaly odkazy ke stažení dat na svých stránkách pouze v národním jazyce (neexistující článek s daty v anglické verzi webu). 
	\begin{table}[H]
	\centering
	\begin{tabular}{|>{\centering\arraybackslash}p{2.2cm}|>{\centering\arraybackslash}p{7cm}|>{\centering\arraybackslash}p{2cm}|>{\centering\arraybackslash}p{1cm}|} 
	\hline
	\textbf{Země}    & \textbf{Patentový úřad} & \textbf{Zkratka} & \textbf{Data}                \\ 
	\hline
	Anglie & \href{https://www.gov.uk/topic/intellectual-property}{Intellectual Property Office}  & IPO & ANO        \\ 
	\hline
	Arménie & \href{https://www.aipa.am/hy/}{Intellectual Property Office}  & - & NE        \\ 
	\hline
	Austrálie & \href{https://www.ipaustralia.gov.au/}{IP Australia}  & - & ANO         \\ 
	\hline
	Bělorusko & \href{https://www.ncip.by/}{National Center of Intellectual Property}  & NCIP & NE         \\ 
	\hline
	Bulharsko & \href{https://www.bpo.bg/}{Patent Office of Republic of Bulgaria}  & -  & NE       \\ 
	\hline
	Česko & \href{https://upv.gov.cz/}{Industrial Property Office of the Czech Republic}  & -   & ANO      \\ 
	\hline
	Čína & \href{https://www.cnipa.gov.cn/}{China National Intellectual Property Administration}  & CNIPA   & NE      \\ 
	\hline
	Dánsko & \href{https://www.dkpto.org/}{Danish Patent and Trademark Office}  & -    & NE     \\ 
	\hline
	Egypt & \href{http://www.egypo.gov.eg}{Egyptian Patent Office}  & -   & NE      \\ 
	\hline
	Estonsko & \href{https://www.epa.ee/et}{The Estonian Patent Office}  & -   & NE      \\ 
	\hline
	Filipíny & \href{http://www.ipophil.gov.ph/}{Intellectual Property Office of the Philippines}  & IPOPHL & NE        \\ 
	\hline
	Finsko & \href{http://www.prh.fi/en/index.html}{Finnish Patent and Registration Office}  & PRH   & NE      \\ 
	\hline
	Francie & \href{http://www.inpi.fr/}{National Institute of Industrial Property}  & INPI   & ANO      \\ 
	\hline
	Hong Kong & \href{https://www.ipd.gov.hk/index.htm}{Intellectual Property Department}  & -   & NE      \\ 
	\hline
	Chorvatsko & \href{https://www.dziv.hr/}{State Intellectual Property Office of the Republic of Croatia}  & SIPO  & NE       \\ 
	\hline
	Indie & \href{http://www.ipindia.nic.in/}{Office of the Controller General of Patents, Designs and Trade Marks}  & -    & NE    \\ 
	\hline
	Indonésie & \href{http://www.dgip.go.id/}{Directorate General of Intellectual Property}  & DGIP & NE        \\ 
	\hline
	Irsko & \href{https://www.ipoi.gov.ie/en/}{Intellectual Property Office of Ireland}  & IPOI   & NE      \\ 
	\hline
	Island & \href{https://www.isipo.is/}{Icelandic Intellectual Property Office}  & ISIPO    & NE     \\ 
	\hline
	Israel & \href{https://www.gov.il/en/departments/ilpo}{The Israel Patent Office}  & ILPO    & ANO     \\ 
	\hline
	Itálie & \href{https://uibm.mise.gov.it/index.php/it/}{Directorate General for the Protection of Industrial Property}  & -    & ANO*     \\ 
	\hline
	Japonsko & \href{https://www.jpo.go.jp/e/index.html}{Japan Patent Office}  & JPO  & ANO       \\ 
	\hline
	Jižní Korea & \href{http://www.kipo.go.kr/}{Korean Intellectual Property Office}  & KIPO   & ANO      \\ 
	\hline	
	Kanada & \href{https://www.ic.gc.ca/}{Canadian Intellectual Property Office}  & CIPO  & ANO      \\ 
	\hline
	\end{tabular}
	\caption{Národní patentové úřady a jejich zkratky, část první.}
	\label{tab:table_offices1}
	\end{table}
\newpage
	\begin{table}[H]
	\centering
	\begin{tabular}{|>{\centering\arraybackslash}p{2.2cm}|>{\centering\arraybackslash}p{7cm}|>{\centering\arraybackslash}p{2cm}|>{\centering\arraybackslash}p{1cm}|} 
	\hline
	\textbf{Země}    & \textbf{Patentový úřad} & \textbf{Zkratka}        & \textbf{Data}        \\ 
	\hline
	Kuba & \href{http://www.ocpi.cu}{Cuban Industrial Property Office}  & OCPI   & NE      \\ 
	\hline
	Litva & \href{http://vpb.lrv.lt/en/}{State Patent Bureau of the Republic of Lithuania}  & -    & ANO     \\
	\hline
 	Lotyšsko & \href{https://www.lrpv.gov.lv/lv}{Patent Office of the Republic of Latvia}  & -    & NE     \\ 
	\hline
	Maďarsko & \href{http://www.hipo.gov.hu/}{Hungarian Intellectual Property Office}  & HIPO   & NE      \\ 
	\hline
	Malajsie & \href{http://www.myipo.gov.my/}{Intellectual Property Corporation of Malaysia}  & MyIPO  & NE       \\ 
	\hline
	Mexiko & \href{https://www.gob.mx/impi/en}{Instituto Mexicano De La Propiedad Industrial}  & IMPI  & ANO       \\ 
	\hline
	Moldova & \href{http://www.agepi.gov.md/}{State Agency on Intellectual Property}  & AGEPI   & NE      \\ 
	\hline
	Německo & \href{http://www.dpma.de/}{German Patent and Trade Mark Office}  & DPMA   & ANO      \\ 
	\hline
	Nizozemsko & \href{http://www.rvo.nl/octrooien}{Netherlands Patent Office}  & -    & NE    \\ 
	\hline
	Norsko & \href{https://www.patentstyret.no/en/}{Norwegian Industrial Property Office}  & NIPO    & NE     \\ 
	\hline
	Nový Zéland & \href{http://www.iponz.govt.nz/}{Intellectual Property Office of New Zealand}  & IPONZ   & ANO      \\ 
	\hline
	Peru & \href{http://www.indecopi.gob.pe/}{National Institute for the Defense of Competition and Protection of Intellectual Property}  & INDECOPI    & ANO     \\ 
	\hline
	Polsko & \href{https://uprp.gov.pl/pl}{Urząd Patentowy Rzeczypospolitej Polskiej}  & UPRP    & ANO     \\ 
	\hline
	Portugalsko & \href{https://inpi.justica.gov.pt/}{Portuguese Institute of Industrial Property}  & -     & ANO    \\ 
	\hline
	Rakousko & \href{http://www.patentamt.at/}{Austrian Patent Office}  & -     & NE    \\ 
	\hline
	Rumunsko & \href{http://www.osim.ro/}{State Office for Inventions and Trademarks}  & OSIM      & NE   \\ 
	\hline
	Rusko & \href{https://rospatent.gov.ru/}{Federal Service for Intellectual Property}  & Rospatent   & ANO      \\ 
	\hline
	Řecko & \href{http://www.obi.gr/el/}{Hellenic Industrial Property Organization}  & HIPO   & NE      \\ 
	\hline
	Singapur & \href{http://www.ipos.gov.sg/}{Intellectual Property Office of Singapore}  & IPOS    & NE     \\ 
	\hline
	Slovensko & \href{https://www.indprop.gov.sk/}{Industrial Property Office of the Slovak Republic}  & -     & NE    \\ 
	\hline
	Slovinsko & \href{http://www.uil-sipo.si/}{Slovenian Intellectual Property Office}  & SIPO   & NE      \\ 
	\hline
	Srbsko & \href{http://www.zis.gov.rs/}{Intellectual Property Office of the Republic of Serbia}  & -   & NE      \\ 
	\hline
	Španělsko & \href{http://www.oepm.es/}{Spanish Patent and Trademark Office}  & OEPM   & ANO      \\ 
	\hline
	Švédsko & \href{http://www.prv.se/}{Swedish Intellectual Property Office}  & PRV   & ANO      \\ 
	\hline
	Švýcarsko & \href{https://www.ige.ch/}{Swiss Federal Institute of Intellectual Property}  & -     & NE    \\ 
	\hline
	Turecko & \href{http://www.turkpatent.gov.tr/}{Turkish Patent and Trademark Office}  & Turkpatent   & NE      \\ 
	\hline
	Ukrajina & \href{https://ukrpatent.org/en}{Ukrainian Intellectual Property Institute}  & Ukrpatent    & NE     \\ 
	\hline
	\end{tabular}
	\caption{Národní patentové úřady a jejich zkratky, část druhá.}
	\label{tab:table_offices2}
	\end{table}
\newpage

\noindent Z celkových 51 patentových zdrojů nám pouze 19 zdrojů poskytuje data. V případě Itálie nám data neposkytuje přímo patentový úřad, ale výzkumný úřad PATIRIS, který poskytuje patentová data z univerzit a veřejných výzkumných ústavů v Itálii. 
\newline
\indent Bohužel ne všechny patentové úřady poskytují svá data zdarma. Celkem tři úřady - Austrálie, Německo, Nový Zéland chtěli za svá data zaplatit. 
\newline
\indent Ještě je potřeba zmínit Japonsko, které svá data poskytuje, ale je potřeba vyplnit formulář, ve kterém bylo potřeba naskenovat oficiální dokument potvrzující adresu školy. Z tohoto důvodu jsme bylo Japonsko jako jako zdroj dat zavrhnuto. Z původních 19 zdrojů dat poskytující data zůstalo nakonec jen 14 zdrojů dat, které poskytují svá data zadarmo.

\subsection{Atributy}\label{subsec:atributy}
Při zadávání práce byly definovány podmínky pro výběr platných zdrojů dat, a jednou z nich bylo i povinnost mít důležité atributy ve struktuře dat patentu. Celkem byly definovány čtyři povinné atributy společně s osmi nepovinnými. Povinné a nepovinné atributy jsou podrobně popsány níže.
\subsubsection{Povinné atributy}
Se zadavatelem bylo domluveno, že validní zdroj patentových dat musí poskytovat patenty obsahující tyto atributy:
\begin{itemize}
\item \textbf{Titulek} - Titulek patentu, který říká o čem daný patent.
\item \textbf{Rok přihlášky / publikace} - Patent musí obsahovat alespoň rok přihlášky / publikace. Publikace / přihláška musí být minimálně z roku 2000, patenty před rokem 2000 budou zamítnuty.
\item \textbf{Autor} - U patentu bude nutné vědět jeho autor (jméno autora, případně název instituce).
\item \textbf{ID patentu} - Patent musí mít nějaké kódové označení / identifikátor, podle kterého ho lze vyhledávat. Identifikátor se bude držet formátu viz obrázek č. \ref{fig:patent_id}, ale nemusí obsahovat všechny položky, protože každá země může některé položky zanedbávat, případně měnit počet znaků v položce, viz tabulka č. \ref{tab:patent_table_id}.
\end{itemize}

	\begin{figure}[H]
	\centering
	\includegraphics[width=12cm]{img/patent_id}
	\caption{Základní formát pro identifikátor patentu \cite{patent_id_format}.}
	\label{fig:patent_id}
	\end{figure}

	\begin{table}[H]
	\centering
	\begin{tabular}{|>{\centering\arraybackslash}p{2.2cm}|>{\centering\arraybackslash}p{1cm}|>{\centering\arraybackslash}p{2cm}|>{\centering\arraybackslash}p{2cm}|>{\centering\arraybackslash}p{3cm}|>{\centering\arraybackslash}p{1cm}|}
	\hline
	\textbf{Země}    & \textbf{CC} & \textbf{TT} & \textbf{YYYY} & \textbf{NNNNNNNN} & \textbf{KK} \\
	\hline
	Austrálie & AU & & 4 znaky & 6 znaků & ANO \\
	\hline
	Kanada & CA & & & 7 znaků & ANO \\
	\hline
	Čína & CN & 1 znak & & 8 znaků & ANO \\
	\hline
	EPO & EP & & & 7 znaků & ANO \\
	\hline
	Německo & DE & 2 znaky & 4 znaky & 6 znaků & ANO \\
	\hline
	Francie & FR & & & 7 znaků & ANO \\
	\hline
	Velká Británie & GB & & & 7 znaků & ANO \\
	\hline
	Nizozemsko & NL & & & 7 znaků & ANO \\
	\hline
	Japonsko & JP & & 4 znaky & 6 znaků & ANO \\
	\hline
	Korea & KR & 2 znaky & 4 znaky & 7 znaků & ANO \\
	\hline
	Rusko & RU & & 4 znaky & 6 znaků & ANO \\
	\hline
	USA & US & & 4 znaky & 7 znaků & ANO \\
	\hline
	WIPO & PCT & & 4 znaky & 6 znaků & ANO \\
	\hline
	\end{tabular}
	\caption{Aktuálně používané formáty pro patenty z různých zemí \cite{patent_id_format}.}
	\label{tab:patent_table_id}
	\end{table}

\noindent V tabulce č. \ref{tab:table_attributes_critical} lze vidět patenty, poskytované národními patentovými úřady zdarma, obsahují povinné atributy.
	\begin{table}[H]
	\centering
	\begin{tabular}{|>{\centering\arraybackslash}p{2.2cm}|>{\centering\arraybackslash}p{2cm}|>{\centering\arraybackslash}p{3cm}|>{\centering\arraybackslash}p{2cm}|>{\centering\arraybackslash}p{2.5cm}|} 
	\hline
	\textbf{Země}    & \textbf{Titulek patentu} & \textbf{Rok přihlášky / publikace} & \textbf{Autor} & \textbf{ID patentu}                \\ 
	\hline
	Kanada & x & x & x & x \\
	\hline
	Česko & x & x & - & x \\
	\hline
	Litva & x & x & x & x \\
	\hline
	Portugalsko & x & x & x & x \\
	\hline
	Španělsko & x & x & x & x \\
	\hline
	Švédsko & - & x & - & x \\
	\hline
	Izrael & x & x & x & x \\
	\hline
	Itálie & x & x & x & x \\
	\hline
	Mexiko & x & x & x & x \\
	\hline
	Polsko & x & x & - & - \\
	\hline
	Anglie & x & x & x & x \\
	\hline
	Rusko & x & x & x & x \\
	\hline
	Peru & x & x & x & x \\
	\hline
	Francie & x & x & x & x \\
	\hline
	\end{tabular}
	\caption{Povinné atributy nacházející se v dostupných patentech.}
	\label{tab:table_attributes_critical}
	\end{table}

\subsubsection{Nepovinné atributy}
Při průzkumu byly zjišťovány i nepovinné atributy, které nemají vliv na výběr zdrojů dat, ale je dobré vědět co který patent z daného patentového zdroje poskytuje za atributy. Nepovinné atributy jsou:
\begin{itemize}
\item \textbf{Abstrakt} - Stručný výtah patentu, který popisuje o čem daný patent je.
\item \textbf{Klíčová slova} - Klíčová sloa nebo fráze spojené s patentem. Mohou sloužit při vyhledávání patentů se stejným zaměřením.
\item \textbf{Reference} - Reference na podobné typy patentů nebo na související patenty (například odkaz na základní verzi algoritmu).
\item \textbf{Žadatel} - Žadatel a autor může být tatáž osoba, ale v některých případech je žadatelem někdo jiný (například autor je zaměstnanec firmy, žadatelem je samotná firma).
\item \textbf{Adresa autora / instituce} - Adresa autora nebo instituce.
\item \textbf{Rodina patentů} - Rodina patentů je kolekce patentových žádostí, které se zaměřují na stejný nebo alespoň podobný technický obsah.
\item \textbf{Obor} - Obor, který daný patent pokrývá. Obory jsou klasifikovány podle systému \gls{IPC} používaného ve více než 100 zemích. Seznam základních oborů lze vidět v tabulce č. \ref{tab:kind_codes}, úplnou klasifikaci lze vidět na stránkách \gls{WIPO}.
\item \textbf{Full-text} - Zdroje dat poskytují veškerá data o patentu (nejenom to co je v přihláškách / publikacích, například různé poznámky, obrázky).
\end{itemize}

	\begin{table}[H]
	\centering
	\begin{tabular}{|>{\centering\arraybackslash}p{1cm}|>{\centering\arraybackslash}p{12cm}|}
	\hline
	\textbf{Kód}    & \textbf{Popisek}\\
	\hline
	A & Lidské potřeby - jídlo, léky, oblečení, ... \\
	\hline
	B & Operace a Doprava - tisk, auta, koleje, nanotechnologie, ...\\
	\hline
	C & Chemie a Hutnictví - sklo, cement, železo, ...\\
	\hline
	D & Textílie a Papír - výroba papíru, provazy, tkalcovství \\
	\hline
	E & Pevné konstrukce -  stavby, dveře, zámky, dolování, ... \\
	\hline
	F & Strojírenství, Osvětlení, Vytápění, Zbraně, Odstřelování \\
	\hline
	G & Fyzika -  měření, testování, optika, nukleární fyzika, ...\\
	\hline
	H & Elektřina - zákadní elektrické elementy (kabely, rezistory, ...), techniky elektrické komunikace, ...  \\
	\hline
	\end{tabular}
	\caption{Základní \gls{IPC} klasifikace oborů \cite{ipc_class}.}
	\label{tab:kind_codes}
	\end{table}

\noindent V tabulkách č. \ref{tab:table_attributes_notcrit1}, č. \ref{tab:table_attributes_notcrit2} lze vidět patenty, poskytované národními patentovými úřady zdarma, obsahující nepovinné atributy.
	\begin{table}[H]
	\centering
	\begin{tabular}{|c|c|c|c|c|} 
	\hline
	\textbf{Země}    & \textbf{Abstrakt} & \textbf{Klíčová slova} & \textbf{Reference} & \textbf{Žadatel} \\
	\hline
	Kanada & - & - & - & x \\
	\hline
	Česko & x & - & x & - \\
	\hline
	Litva & x & - & - & x \\
	\hline
	Portugalsko & x & - & - & x \\
	\hline
	Španělsko & x & - & - & x \\
	\hline
	Švédsko & x & - & - & - \\
	\hline
	Izrael & - & - & - & x \\
	\hline
	Itálie & - & - & - & x \\
	\hline
	Mexiko & x & - & - & x \\
	\hline
	Polsko & x & x & - & - \\
	\hline
	Anglie & - & - & - & - \\
	\hline
	Rusko & - & - & - & - \\
	\hline
	Peru & - & - & - & - \\
	\hline
	Francie & x & - & x & x \\
	\hline
	\end{tabular}
	\caption{Nepovinné atributy nacházející se v dostupných patentech, část první.}
	\label{tab:table_attributes_notcrit1}
	\end{table}

	\begin{table}[H]
	\centering
	\begin{tabular}{|c|c|c|c|c|} 
	\hline
	\textbf{Země}    &  \textbf{Adresa} & \textbf{Rodina patentů} & \textbf{Obor} & \textbf{Full-text} \\
	\hline
	Kanada & x & - & x & - \\
	\hline
	Česko & - & - & x & - \\
	\hline
	Litva & - & - & x & - \\
	\hline
	Portugalsko & - & - & x & - \\
	\hline
	Španělsko & x & - & x & x \\
	\hline
	Švédsko & - & - & x & x \\
	\hline
	Izrael & x & - & - & - \\
	\hline
	Itálie & - & - & - & - \\
	\hline
	Mexiko & - & - & x & - \\
	\hline
	Polsko & - & - & - & - \\
	\hline
	Anglie & - & - & x & - \\
	\hline
	Rusko & - & - & - & - \\
	\hline
	Peru & - & - & x & - \\
	\hline
	Francie & - & - & x & - \\
	\hline
	\end{tabular}
	\caption{Nepovinné atributy nacházející se v dostupných patentech, část druhá.}
	\label{tab:table_attributes_notcrit2}
	\end{table}


\subsection{Závěr průzkumu}
Průzkum národních zdrojů zahrnoval celkem 51 národních patentujících institucí, ze kterých pouze 10 poskytovalo svá data zdarma a splňovala všechny podmínky. V tabulce č. \ref{tab:patent_rozdeleni} lze vidět souhrn výsledků.

	\begin{table}[H]
	\centering
	\begin{tabular}{|>{\centering\arraybackslash}p{3cm}|>{\centering\arraybackslash}p{2cm}|>{\centering\arraybackslash}p{2.2cm}|}
	\hline
	\textbf{Popis}    & \textbf{Počet} & \textbf{Počet v \%}\\
	\hline
	Nedostupné & 33 & 64,70 \%\\
	\hline
	Nepoužitelné & 4 & 7,85 \%\\
	\hline
	Za peníze & 4 & 7,85 \%\\
	\hline
	Použitelné & 10 & 19,60 \%\\
	\hline
	\end{tabular}
	\caption{Souhrn průzkumu národních patentujících institucí.}
	\label{tab:patent_rozdeleni}
	\end{table}

\noindent Pro všechny použitelné národní zdroje bylo kromě výskytu atributů dále sledováno: formát uložených dat, počet patentů po roce 2000 (včetně roku 2000) obsahující všechny povinné atributy, počet patentů před rokem 2000 a počet duplikátů. Duplikátem se myslí jiná verze daného patentu, protože v průběhu let se mohl měnit obsah patentu a v databázi nechceme ukládat žádné starší verze jednoho patentu (výsledky vyhledávání v databázi nebudou validní). V tabulce č. \ref{tab:final_zdroje} lze vidět všechny validní národní zdroje.

\todo{Dopsat itálii po přidání do mysql  + souhrn přepsat}

	\begin{table}[H]
	\centering
	\begin{tabular}{|>{\centering\arraybackslash}p{2.2cm}|>{\centering\arraybackslash}p{1.5cm}|>{\centering\arraybackslash}p{3cm}|>{\centering\arraybackslash}p{3cm}|>{\centering\arraybackslash}p{2.2cm}|}
	\hline
	\textbf{Země}    & \textbf{Formát dat} & \textbf{Počet patentů (rok >= 2000)} & \textbf{Počet patentů (rok < 2000)} & \textbf{Počet duplikátů}\\
	\hline
	Anglie & XML & 88 032 & 141 & 19\\
	\hline
	Francie & XML & 273 193 & 192 630 & 1 140 084\\
	\hline
	Israel & XML & 116 373 & 0 & 9 956\\
	\hline
	Itálie & SQL & 17 622 & 3 728 & 0\\
	\hline
	Kanada & XML & 816 828 & 130 279 & 499 682\\
	\hline
	Litva & XML & 869 & 0 & 0\\
	\hline
	Peru & XLSX & 1 805  & 21 352 & 0\\
	\hline
	Portugalsko & XML & 69 & 0 & 0\\
	\hline
	Rusko & CSV & 614 035 & 139 714 & 15 166 816\\
	\hline
	Španělsko & XML & 325 662  & 37 596 & 39 033\\
	\hline
	&&&& \\
	\hline
	\textbf{Souhrn} & & \textbf{2 236 866} & \textbf{525 440} & \textbf{16 855 590} \\
	\hline
	\end{tabular}
	\caption{Seznam všech validních národních zdrojů.}
	\label{tab:final_zdroje}
	\end{table}

\section{Výběr databáze}
V kapitole č. \ref{chap:databaze} bylo podrobně popsáno co databáze je, jaké typy databází dnes existují (krátký výčet) i nejznámější existující řešení pro popsané typy databází. V této kapitole budou podrobně popsány rozdíly mezi jednotlivými řešeními a následně se vybere nejlepší typ databáze pro ukládání velkého objemu patentových dat. Následně, podle vybraného typu databáze, se vybere nejvhodnější existující řešení.

\subsection{Výběr typu databáze}
Abychom zajistili co nejefektivnější vytěžování, tak je potřeba vybrat co nejvhodnější typ databáze vzhledem k povaze úlohy. V budoucnu se očekává, že počet skladovaných patentů bude v řádech jednotek až desítek milionů (nelze vyvrátit i stovky milionů v případě, že se budou ukládat i patenty z jiných než národních zdrojů).

\subsubsection{Relační databáze}
Relační databáze není vhodným kandidátem pro ukládání nestrukturovaných patentových dat. V databázi sice existuje datový typ \textbf{BLOB}, který umožňuje ukládat binární soubory (v našem případě soubor s patentovými daty), ale nelze to pokládat za nejlepší řešení, když existují například dokumentové databáze. Lze zmínit i datový typ \textbf{TEXT} / \textbf{LONGTEXT}, který umožňuje ukládát velké množství textu a lze ho procházet pomocí full-text vyhledávání, ale výkonnostně a rychlostně se stejně nevyrovná NoSQL databázím. 
\newline
\indent Využití relační databáze by mělo smysl pouze v případě vytváření statistik (například počet v patentů v Kanadě za rok 2020). Tento přístup by ale vyžadoval extrahovat specifická data (například jen povinné atributy) ze souboru pomocí parseru a následné uložení hodnot do tabulek. 

\subsubsection{Objektově-orientovaná databáze}
Objektově-orientovaná databáze, stejně jako relační databáze, není vhodným kandidátem pro ukládání nestrukturovaných dat. Lze argumentovat vytvořením objektů odpovídající struktuře dokumentu, ale při vložení dokumentu s jinou strukturou nastává problém s uložením atributů, které se nenachází v objektu. V některých případech může vysoká složitost systému zpomalovat vyhledávání. Velká výhoda objektově-orientované datábaze spočívá v jednoduchém mapování objektů při práci s objektově-orientovaném programování, které ale v našem případě nemá využití. V případě vytěžování statistik je objektově-orientovaná databáze horší volbou než relační databáze.

\subsubsection{Databáze klíč-hodnota}
Databáze klíč-hodnota je jednoduchá a velice rychlá databáze, která umožňuje ukládat i nestrukturovaná data a nepotřebuje k tomu velké množství paměti. Její nevýhoda je ale v ukládání složitých struktur, které soubory s patenty mají. Lze uložit celou strukturu patentu jako hodnotu, ale následné vyhledávání hodnot pomocí názvů parametrů je nemožné. Použití databáze klíč-hodnota v našem případě není moc vhodné.

\subsubsection{Grafová databáze}
Grafová databáze není vhodným kandidátem pro ukládání patentů, protože se zaměřuje hlavně na vztahy mezi jednotlivými daty, což u patentů nelze a ani není potřeba sledovat.

\subsubsection{Databáze dokumentů}
Databáze dokumentů, jak už název napovídá, je databáze pro efektivní ukládání dokumentů a jejich vytěžování. Umožňuje ukládat velké množství nestrukturovaných dat, její udržba je snadná a akceptuje dokumenty v několika datových formátech. Její velká nevýhoda je v kontrole konzistence, takže se v databázi mohou vyskytovat duplikáty. I přes tuto nevýhodu je dokumentová databáze vhodným kandidátem pro ukládání patentových dat.

\subsubsection{Závěr}
Dokumentová databáze bude použita jako primární databáze, protože umožňuje ukládat nestrukturované dokumenty velice efektivně. Zároveň podporuje vkládání dokumentů ve více formátech, což v případě mnoha národních zdrojů, kdy každý zdroj ukládá dat v jiném formátu, je velice vhodná vlastnost. Její nevýhoda, kontrola konzistence, může být odstraněna pomocí jednoduché aplikace, která bude kontrolovat výskyt patentu v databázi podle jeho identifikátoru (ID). Dokumentová databáze podporuje i full-text vyhledávání pro efektivnější a rychlejší vyhledávání.
\newline
\indent Relační databáze bude použita jako sekundární databáze pro vytěžování, zaměřená hlavně na tvorbu statistik. Relační databáze je vhodnou volbou pro získávání statistik, protože vyhledávání je velice rychlé a snadné. \gls{SQL} dotazy pro statistiky se můžou uložit do pohledů, které zjednoduší uživateli práci se zadáváním dotazů. Lze zmínit i nevýhody jako třeba problém s údržbou nebo potřeba velkého množství paměti. Tyto nevýhody ale nehrají velkou roli v případě jednotek až desítek milionů záznamů.

\subsection{Výběr z existujících řešení}
V dnešní době existuje mnoho dokumentových i relačních databází, placených i zdarma poskytovaných. Placené verze oproti verzím zdarma mají výhodu v lepší podpoře ze strany vývojářů, obsahují více užitečných funkcí a mají lepší zabezpečení. Pro naše účely bohatě postačí verze zdarma.
\newline
\indent Jako existující řešení databáze dokumentů byla vybrána komunitní verze databáze MongoDB. Komunitní verze je zdarma a server lze provozovat jak lokálně, tak i na cloudu, kde MongoDB poskytuje zdarma uložiště o velikosti 512 MB. Pro lepší a spolehlivější vyhledávání v datech bude MongoDB spojena s vyhledávačem Elasticsearch. Elasticsearch je full-textový open-source vyhledávač, který nabízí vysokou dostupnost, rychlost a škálovatelnost. MongoDB sice obsahuje vlastní full-textový vyhledávač, který ale není tak výkonný jako Elasticsearch.
\newline
\indent Jako existující řešení relační databáze byla vybrána komunitní verze databáze MySQL. MySQL je skvělá databáze, která se používá hlavně pro čtení dat. Zároveň je to jedna z nejpoužívanějších relačních databází, což znamená, že je pro ní k dispozici více nástrojů třetích stran.
%%CHAPTER
\chapter{Návrh struktury}
% Co a jak bude propojeno ( + diagram)
% Import dat
%CHAPTER
\chapter{Implementace úložiště}
\todo{todo}

\section{Adresářová struktura}
\subsection{Patenty}
\subsection{Docker}

\section{Výběr dat}
%CHAPTER
\chapter{Výběr dat}
Zdroje (které ano, které ne + proč) + udělat stejnou tabulku jak v excelu + zmínit stránku ze který jsem čerpal informace (wipo.int)
	\begin{table}[h!]
	\centering
	\begin{tabular}{|>{\centering\arraybackslash}p{2.2cm}|>{\centering\arraybackslash}p{8cm}|>{\centering\arraybackslash}p{2cm}|} 
	\hline
	\textbf{Země}    & \textbf{Patentový úřad} & \textbf{Zkratka}                \\ 
	\hline
	Anglie & \href{https://www.gov.uk/topic/intellectual-property}{Intellectual Property Office}  & IPO         \\ 
	\hline
	Arménie & \href{https://www.aipa.am/hy/}{Intellectual Property Office}  & -         \\ 
	\hline
	Austrálie & \href{https://www.ipaustralia.gov.au/}{IP Australia}  & -         \\ 
	\hline
	Bělorusko & \href{https://www.ncip.by/}{National Center of Intellectual Property}  & NCIP         \\ 
	\hline
	Bulharsko & \href{https://www.bpo.bg/}{Patent Office of Republic of Bulgaria}  & -         \\ 
	\hline
	Česko & \href{https://upv.gov.cz/}{Industrial Property Office of the Czech Republic}  & -         \\ 
	\hline
	Čína & \href{https://www.cnipa.gov.cn/}{China National Intellectual Property Administration}  & CNIPA         \\ 
	\hline
	Dánsko & \href{https://www.dkpto.org/}{Danish Patent and Trademark Office}  & -         \\ 
	\hline
	Egypt & \href{http://www.egypo.gov.eg}{Egyptian Patent Office}  & -         \\ 
	\hline
	Estonsko & \href{https://www.epa.ee/et}{The Estonian Patent Office}  & -         \\ 
	\hline
	Filipíny & \href{http://www.ipophil.gov.ph/}{Intellectual Property Office of the Philippines}  & IPOPHL         \\ 
	\hline
	Finsko & \href{http://www.prh.fi/en/index.html}{Finnish Patent and Registration Office}  & PRH         \\ 
	\hline
	Francie & \href{http://www.inpi.fr/}{National Institute of Industrial Property}  & INPI         \\ 
	\hline
	Hong Kong & \href{https://www.ipd.gov.hk/index.htm}{Intellectual Property Department}  & -         \\ 
	\hline
	Chorvatsko & \href{https://www.dziv.hr/}{State Intellectual Property Office of the Republic of Croatia}  & SIPO         \\ 
	\hline
	Indie & \href{http://www.ipindia.nic.in/}{Office of the Controller General of Patents, Designs and Trade Marks}  & -         \\ 
	\hline
	Indonésie & \href{http://www.dgip.go.id/}{Directorate General of Intellectual Property}  & DGIP         \\ 
	\hline
	Irsko & \href{https://www.ipoi.gov.ie/en/}{Intellectual Property Office of Ireland}  & IPOI         \\ 
	\hline
	Island & \href{https://www.isipo.is/}{Icelandic Intellectual Property Office}  & ISIPO         \\ 
	\hline
	Israel & \href{https://www.gov.il/en/departments/ilpo}{The Israel Patent Office}  & ILPO         \\ 
	\hline
	Itálie & \href{https://uibm.mise.gov.it/index.php/it/}{Directorate General for the Protection of Industrial Property}  & -         \\ 
	\hline
	Japonsko & \href{https://www.jpo.go.jp/e/index.html}{Japan Patent Office}  & JPO         \\ 
	\hline
	Jižní Korea & \href{http://www.kipo.go.kr/}{Korean Intellectual Property Office}  & KIPO         \\ 
	\hline	
	Kanada & \href{https://www.ic.gc.ca/}{Canadian Intellectual Property Office}  & CIPO         \\ 
	\hline
	\end{tabular}
	\caption{Národní patentové úřady a jejich zkratky, část první}
	\label{tab:table_offices1}
	\end{table}

\newpage

	\begin{table}[h!]
	\centering
	\begin{tabular}{|>{\centering\arraybackslash}p{2.2cm}|>{\centering\arraybackslash}p{8cm}|>{\centering\arraybackslash}p{2cm}|} 
	\hline
	\textbf{Země}    & \textbf{Patentový úřad} & \textbf{Zkratka}                \\ 
	\hline
	Kuba & \href{http://www.ocpi.cu}{Cuban Industrial Property Office}  & OCPI         \\ 
	\hline
	Litva & \href{http://vpb.lrv.lt/en/}{State Patent Bureau of the Republic of Lithuania}  & -         \\
	\hline
 	Lotyšsko & \href{https://www.lrpv.gov.lv/lv}{Patent Office of the Republic of Latvia}  & -         \\ 
	\hline
	Maďarsko & \href{http://www.hipo.gov.hu/}{Hungarian Intellectual Property Office}  & HIPO         \\ 
	\hline
	Malajsie & \href{http://www.myipo.gov.my/}{Intellectual Property Corporation of Malaysia}  & MyIPO         \\ 
	\hline
	Mexiko & \href{https://www.gob.mx/impi/en}{Instituto Mexicano De La Propiedad Industrial}  & IMPI         \\ 
	\hline
	Moldova & \href{http://www.agepi.gov.md/}{State Agency on Intellectual Property}  & AGEPI         \\ 
	\hline
	Německo & \href{http://www.dpma.de/}{German Patent and Trade Mark Office}  & DPMA         \\ 
	\hline
	Nizozemsko & \href{http://www.rvo.nl/octrooien}{Netherlands Patent Office}  & -         \\ 
	\hline
	Norsko & \href{https://www.patentstyret.no/en/}{Norwegian Industrial Property Office}  & NIPO         \\ 
	\hline
	Nový Zéland & \href{http://www.iponz.govt.nz/}{Intellectual Property Office of New Zealand}  & IPONZ         \\ 
	\hline
	Peru & \href{http://www.indecopi.gob.pe/}{National Institute for the Defense of Competition and Protection of Intellectual Property}  & INDECOPI         \\ 
	\hline
	Polsko & \href{https://uprp.gov.pl/pl}{Urząd Patentowy Rzeczypospolitej Polskiej}  & UPRP         \\ 
	\hline
	Portugalsko & \href{https://inpi.justica.gov.pt/}{Portuguese Institute of Industrial Property}  & -         \\ 
	\hline
	Rakousko & \href{http://www.patentamt.at/}{Austrian Patent Office}  & -         \\ 
	\hline
	Rumunsko & \href{http://www.osim.ro/}{State Office for Inventions and Trademarks}  & OSIM         \\ 
	\hline
	Rusko & \href{https://rospatent.gov.ru/}{Federal Service for Intellectual Property}  & Rospatent         \\ 
	\hline
	Řecko & \href{http://www.obi.gr/el/}{Hellenic Industrial Property Organization}  & HIPO         \\ 
	\hline
	Singapur & \href{http://www.ipos.gov.sg/}{Intellectual Property Office of Singapore}  & IPOS         \\ 
	\hline
	Slovensko & \href{https://www.indprop.gov.sk/}{Industrial Property Office of the Slovak Republic}  & -         \\ 
	\hline
	Slovinsko & \href{http://www.uil-sipo.si/}{Slovenian Intellectual Property Office}  & SIPO         \\ 
	\hline
	Srbsko & \href{http://www.zis.gov.rs/}{Intellectual Property Office of the Republic of Serbia}  & -         \\ 
	\hline
	Španělsko & \href{http://www.oepm.es/}{Spanish Patent and Trademark Office}  & OEPM         \\ 
	\hline
	Švédsko & \href{http://www.prv.se/}{Swedish Intellectual Property Office}  & PRV         \\ 
	\hline
	Švýcarsko & \href{https://www.ige.ch/}{Swiss Federal Institute of Intellectual Property}  & -         \\ 
	\hline
	Turecko & \href{http://www.turkpatent.gov.tr/}{Turkish Patent and Trademark Office}  & Turkpatent         \\ 
	\hline
	Ukrajina & \href{https://ukrpatent.org/en}{Ukrainian Intellectual Property Institute}  & Ukrpatent         \\ 
	\hline
	\end{tabular}
	\caption{Národní patentové úřady a jejich zkratky, část druhá}
	\label{tab:table_offices2}
	\end{table}

\newpage

\section{Atributy}
\subsection{Povinné atributy}

	\begin{table}[h!]
	\centering
	\begin{tabular}{|>{\centering\arraybackslash}p{2.2cm}|>{\centering\arraybackslash}p{2cm}|>{\centering\arraybackslash}p{3cm}|>{\centering\arraybackslash}p{2cm}|>{\centering\arraybackslash}p{2.5cm}|} 
	\hline
	\textbf{Země}    & \textbf{Název patentu} & \textbf{Rok přihlášky / patentu} & \textbf{Autor} & \textbf{ID patentu}                \\ 
	\hline
	Kanada & x & x & x & x \\
	\hline
	Česko & x & x & - & x \\
	\hline
	Litva & x & x & x & x \\
	\hline
	Portugalsko & x & x & x & x \\
	\hline
	Španělsko & x & x & x & x \\
	\hline
	Švédsko & - & x & - & x \\
	\hline
	Izrael & x & x & x & x \\
	\hline
	Itálie & x & x & x & x \\
	\hline
	Mexiko & x & x & x & x \\
	\hline
	Polsko & x & x & - & - \\
	\hline
	Anglie & x & x & x & x \\
	\hline
	Rusko & x & x & x & x \\
	\hline
	Peru & x & x & x & x \\
	\hline
	Francie & x & x & x & x \\
	\hline
	\end{tabular}
	\caption{Povinné atributy nacházející se v dostupných patentech}
	\label{tab:table_attributes_critical}
	\end{table}

\subsection{Nepovinné atributy}

	\begin{table}[h!]
	\centering
	\begin{tabular}{|c|c|c|c|c|} 
	\hline
	\textbf{Země}    & \textbf{Abstrakt} & \textbf{Slovník} & \textbf{Reference} & \textbf{Žadatel} \\
	\hline
	Kanada & - & - & - & x \\
	\hline
	Česko & x & - & x & - \\
	\hline
	Litva & x & - & - & x \\
	\hline
	Portugalsko & x & - & - & x \\
	\hline
	Španělsko & x & - & - & x \\
	\hline
	Švédsko & x & - & - & - \\
	\hline
	Izrael & - & - & - & x \\
	\hline
	Itálie & - & - & - & x \\
	\hline
	Mexiko & x & - & - & x \\
	\hline
	Polsko & x & x & - & - \\
	\hline
	Anglie & - & - & - & - \\
	\hline
	Rusko & - & - & - & - \\
	\hline
	Peru & - & - & - & - \\
	\hline
	Francie & x & - & x & x \\
	\hline
	\end{tabular}
	\caption{Nepovinné atributy nacházející se v dostupných patentech, část první}
	\label{tab:table_attributes_notcrit1}
	\end{table}

	\begin{table}[h!]
	\centering
	\begin{tabular}{|c|c|c|c|c|} 
	\hline
	\textbf{Země}    &  \textbf{Adresa} & \textbf{Rodina patentů} & \textbf{Obor} & \textbf{Fulltext} \\
	\hline
	Kanada & x & - & x & - \\
	\hline
	Česko & - & - & x & - \\
	\hline
	Litva & - & - & x & - \\
	\hline
	Portugalsko & - & - & x & - \\
	\hline
	Španělsko & x & - & x & x \\
	\hline
	Švédsko & - & - & x & x \\
	\hline
	Izrael & x & - & - & - \\
	\hline
	Itálie & - & - & - & - \\
	\hline
	Mexiko & - & - & x & - \\
	\hline
	Polsko & - & - & - & - \\
	\hline
	Anglie & - & - & x & - \\
	\hline
	Rusko & - & - & - & - \\
	\hline
	Peru & - & - & x & - \\
	\hline
	Francie & - & - & x & x \\
	\hline
	\end{tabular}
	\caption{Nepovinné atributy nacházející se v dostupných patentech, část druhá}
	\label{tab:table_attributes_notcrit2}
	\end{table}

Rovnou udělat kapitolu, ve který se aplikují všechny podmínky + se sepíše souhrn počtu patentů, z jakých zemí atp.


\section{Implementace databáze}
\subsection{MySQL}
\subsection{MongoDB}

\section{Výsledné úložiště}
\subsection{Technologické požadavky}
\subsection{Docker}
Images, scripty \newline
Import dat (skripty, data soubory, ...) 
\subsection{Inicializace MySQL}
\subsection{Inicializace MongoDB}
%CHAPTER
\chapter{Rozšiřitelnost modulu}
Zadání diplomové práce sice splněno bylo, ale v blízké budoucnosti mohou být požadavky na modul změněny. Jako příklad lze uvést podporu přidávání nových patentů do databází, zjištění autorů pro české patenty, automatické stahování dat z již ověřených patentových zdrojů. V této kapitole jsou popsány 3 možné návrhy na rozšíření modulu ohledně importu dat do již existujících databází.

\section{Přidávání nových patentů}
Cílem tohoto rožšíření by bylo automatické přidávání patentů z datových souborů jak do MySQL databáze, tak i do Mongo.

Rozšíření by se dalo realizovat jako aplikace ve vyšším programovacím jazyku (např. Java, C), kdy vstupem do aplikace by byl soubor v datovém formátu JSON/XML/CSV a jiné. Vstupní soubor by se následně:
\begin{itemize}
\item převedl na JSON řetězec (v případě že soubor není ve formátu JSON) a vložil do Mongo databáze
\item rozparsoval a extrahovali by se všechny atributy, které se ukládají v MySQL databázi (viz mysql kapitola \todo{TODO})
\end{itemize}

Jelikož je dost časté, že každý národní zdroj dat používá odlišnou strukturu patentu, tak bude potřeba aplikaci neustále upravovat (ať už v rámci přidávání nových zdrojů, nebo v případě změny struktury patentu u již podporovaných zdrojů).

Jako další velký problém lze zmínit extrakci atributů patentu ze souborů. Tím, že různé patentové soubory mají odlišnou strukturu, to znamená hloubku zanoření specifických elementů, jiné názvy elementů, tak bude obtížné naimplementovat řešení extrakce pro všechny soubory. Tento problém by se dal řešit tak, že se vytvoří soubory se slovníkama, které by obsahovali názvy elementů pro daný atribut. Slovníky by se následně použily při extrakci.

\section{Zjišťování autorů pro české patenty}
Český národní patentový úřad poskytuje data o českých patentech, které ale neobsahují autora ani instituci. Pro zjištění autora nebo instituce, která patent registrovala, je nutné použít oficiální vyhledávač. Cílem tohoto rozšíření by bylo vytvořit aplikaci ve vyšším programovacím jazyku, která se pro všechny české patenty bude snažit najít jejich autory za pomoci využití prohledávačů webů (web crawler). Postupů řešení může být mnoho:
\begin{itemize}
\item Zjišťování autorů by se provedlo pro všechny existující české patenty v databázi. Z MySQL databáze se zjistí všechny ID patentů pro české patenty, které se následně použijí jako vstup pro web crawler.
\item Zjišťování autorů by se provedlo pro patent/y uložené v souboru, kdy aplikace by pro všechny patenty v souboru zjistila autory a následně je dopsala do příslušnýho elementu patentu v daném souboru.
\item Stejný postup jako předchozí s tím rozdílem, že po zjištění autora se patent rovnou přidá do MySQL i Mongo databáze.
\end{itemize}

\section{Automatické stahování dat z ověřených zdrojů}
Aplikace, která si načte soubor s uloženýma ověřenýma zdrojema (např. XML), kdy by bylo definováno: název země, URL stránky kde se stahuje (např. https://isdv.upv.cz/webapp/webapp.pubsrv.seznam?purl=opendata/pt), html element obsahující soubor ke stažení (např. a href="/doc/opendata/pt/OpenData....") a poslední stažený soubor. Aplikace by projela všechny stránky, pokusila by se stáhnout nejnovější soubor (ten co se neshoduje s tím uloženým v xml) a uložit ho do specifický složky. Pokud je ke stažení novější, updatuje se XML.

Tohle všechny by bylo automatický pomocí např. Jenkinsu, který by jednou za čas spustil tuhle appku + by mohl informovat emailem o buildu + výsledku, že tolik se updatnulo atd.
%Automatické stahování dat z již ověřených zdrojů - kontrola zda přibyl nový ZIP, pokud ano tak stáhnout, rozbalit - následně získat patent ID a zkontrolovat duplicitní záznam - zahodit / případně nahradit existující kvůli novým změnám

%\lstset{style=sqlstyle}

%CHAPTER
\chapter{Ověření efektivního vytěžování}
K ověření efektivního vytěžování bylo připraveno několik scénářů jak pro SQL, tak i pro Mongo + ElasticSearch.\newline
Napsat referenční stroj na kterém se testovalo - CPU, RAM, ...\todo{todo}
\section{Mongo + ElasticSearch}
\section{MySQL}
Pro MySQL bylo připraveno 10 scénářů, které testují všechny vytvořené tabulky v databázi. Každý scénář obsahuje textový popis, SQL příkaz, rychlost vykonání příkazu a ukázku výsledků.

\subsection{Scénář č.1}
\textbf{Textový popis}: Pět nejčastěji patentujících institucí v Izraeli v roce 2015
\newline
\textbf{SQL}: 
\begin{lstlisting}[language=SQL, breaklines=true, frame=single, label = {lst:elements_a}, captionpos=b]
select count(*), inventors.inventor from inventors left outer join patents on inventors.id_patent = patents.id where YEAR(patents.patent_date) = 2015 and patents.patent_id like '%IL%' group by inventors.inventor order by count(*) desc LIMIT 5;
\end{lstlisting}
\textbf{Rychlost vykonání dotazu}: \todo{TODO}
\newline
\textbf{Výsledek dotazu}:
\begin{figure}[h!]
\centering
\includegraphics[width=8cm]{img/scenare/scenar_1}
\caption{Ukázka výsledku dotazu pro scénář č.1}
\label{fig:scenar1}
\end{figure}


\subsection{Scénář č.2}
\textbf{Textový popis}: Tři nejméně patentované obory v Kanadě od roku 2010
\newline
\textbf{SQL}: 
\begin{lstlisting}[language=SQL, breaklines=true, frame=single, label = {lst:elements_a}, captionpos=b]
select count(*), classification.section from classification left outer join patents on patents.id = classification.id_patent where YEAR(patents.patent_date) >= 2010 and patents.patent_id like '%CA%' group by classification.section order by count(*) asc LIMIT 3;
\end{lstlisting}
\textbf{Rychlost vykonání dotazu}: \todo{TODO}
\newline
\textbf{Výsledek dotazu}:\todo{TODO}
\begin{figure}[h!]
\centering
\includegraphics[width=6cm]{img/scenare/scenar_9}
\caption{Ukázka výsledku dotazu pro scénář č.2}
\label{fig:scenar2}
\end{figure}

\subsection{Scénář č.3}
\textbf{Textový popis}: Nejčastější klasifikace patentu za rok 2008 ve Španělsku
\newline
\textbf{SQL}: 
\begin{lstlisting}[language=SQL, breaklines=true, frame=single, label = {lst:elements_a}, captionpos=b]
select count(*), classification.section, classification.class, classification.subclass from classification left outer join patents on patents.id = classification.id_patent where YEAR(patents.patent_date) = 2008 and patents.patent_id LIKE '%ES%' group by classification.section, classification.class, classification.subclass order by count(*) desc LIMIT 1;
\end{lstlisting}
\textbf{Rychlost vykonání dotazu}: \todo{TODO}
\newline
\textbf{Výsledek dotazu}:
\begin{figure}[h!]
\centering
\includegraphics[width=8cm]{img/scenare/scenar_3}
\caption{Ukázka výsledku dotazu pro scénář č.3}
\label{fig:scenar3}
\end{figure}

\subsection{Scénář č.4}
\textbf{Textový popis}: Autor s největším počtem patentů ze všech zemí
\newline
\textbf{SQL}: 
\begin{lstlisting}[language=SQL, breaklines=true, frame=single, label = {lst:elements_a}, captionpos=b]
select count(*), inventors.inventor from inventors left outer join patents on patents.id = inventors.id_patent group by inventors.inventor order by count(*) desc LIMIT 1;
\end{lstlisting}
\textbf{Rychlost vykonání dotazu}: \todo{TODO}
\newline
\textbf{Výsledek dotazu}:\todo{TODO}
\begin{figure}[h!]
\centering
\includegraphics[width=6cm]{img/scenare/scenar_9}
\caption{Ukázka výsledku dotazu pro scénář č.4}
\label{fig:scenar4}
\end{figure}

\subsection{Scénář č.5}
\textbf{Textový popis}: Nejméně používaný jazyk pro patenty za rok 2003
\newline
\textbf{SQL}: 
\begin{lstlisting}[language=SQL, breaklines=true, frame=single, label = {lst:elements_a}, captionpos=b]
select count(*), patents.language from patents where patents.language not like '%-%' group by patents.language order by count(*) asc LIMIT 1;
\end{lstlisting}
\textbf{Rychlost vykonání dotazu}: \todo{TODO}
\newline
\textbf{Výsledek dotazu}:\todo{TODO}
\begin{figure}[h!]
\centering
\includegraphics[width=6cm]{img/scenare/scenar_9}
\caption{Ukázka výsledku dotazu pro scénář č.5}
\label{fig:scenar5}
\end{figure}

\subsection{Scénář č.6}
\textbf{Textový popis}: Deset Institucí / autorů s patenty pokrývající největší množství oborů
\newline
\textbf{SQL}: 
\begin{lstlisting}[language=SQL, breaklines=true, frame=single, label = {lst:elements_a}, captionpos=b]
select count(distinct classification.section), inventors.inventor from inventors left outer join classification on classification.id_patent = inventors.id_patent where section is not null group by inventors.inventor order by count(distinct classification.section) desc LIMIT 10;
\end{lstlisting}
\textbf{Rychlost vykonání dotazu}: \todo{TODO}
\newline
\textbf{Výsledek dotazu}:\todo{TODO}
\begin{figure}[h!]
\centering
\includegraphics[width=6cm]{img/scenare/scenar_9}
\caption{Ukázka výsledku dotazu pro scénář č.6}
\label{fig:scenar6}
\end{figure}

\subsection{Scénář č.7}
\textbf{Textový popis}: Země s nejvíce patenty od roku 2018
\newline
\textbf{SQL}: 
\begin{lstlisting}[language=SQL, breaklines=true, frame=single, label = {lst:elements_a}, captionpos=b]
select count(*), patents.country from patents where YEAR(patents.patent_date) >= 2018 group by patents.country order by count(*) desc;
\end{lstlisting}
\textbf{Rychlost vykonání dotazu}: \todo{TODO}
\newline
\textbf{Výsledek dotazu}:\todo{TODO}
\begin{figure}[h!]
\centering
\includegraphics[width=6cm]{img/scenare/scenar_9}
\caption{Ukázka výsledku dotazu pro scénář č.7}
\label{fig:scenar7}
\end{figure}

\subsection{Scénář č.8}
\textbf{Textový popis}: Nejvíce používaný jazyk pro patenty ve Francii
\newline
\textbf{SQL}: 
\begin{lstlisting}[language=SQL, breaklines=true, frame=single, label = {lst:elements_a}, captionpos=b]
select count(*), patents.language from patents where patents.patent_id like '%FR%' group by patents.language order by count(*) desc;
\end{lstlisting}
\textbf{Rychlost vykonání dotazu}: \todo{TODO}
\newline
\textbf{Výsledek dotazu}:\todo{TODO}
\begin{figure}[h!]
\centering
\includegraphics[width=6cm]{img/scenare/scenar_9}
\caption{Ukázka výsledku dotazu pro scénář č.8}
\label{fig:scenar8}
\end{figure}

\subsection{Scénář č.9}
\textbf{Textový popis}: Tři nejčastěji patentující instituce / autoři v Anglii v textilním oboru za rok 2013
\newline
\textbf{SQL}: 
\begin{lstlisting}[language=SQL, breaklines=true, frame=single, label = {lst:elements_a}, captionpos=b]
select count(*), inventors.inventor from inventors left outer join patents on patents.id = inventors.id_patent left outer join classification on classification.id_patent = patents.id where classification.section like '%D%' and patents.patent_id like '%GB%' and YEAR(patents.patent_date) = 2013 group by inventors.inventor order by count(*) desc LIMIT 3
\end{lstlisting}
\textbf{Rychlost vykonání dotazu}: \todo{TODO}
\newline
\textbf{Výsledek dotazu}:
\begin{figure}[h!]
\centering
\includegraphics[width=6cm]{img/scenare/scenar_9}
\caption{Ukázka výsledku dotazu pro scénář č.9}
\label{fig:scenar9}
\end{figure}
%CHAPTER
\chapter{Závěr}
V rámci této práce se autor seznámil s dostupnými zdroji dat o patentech. Celosvětové patentové instituce nebyly prostudovány z důvodu již existující diplomové práce \cite{BARATTA2019thesis}, která byla zaměřená právě na tyto celosvětové instituce. Tato diplomová práce rozšiřuje původní diplomovou práci ve směru národních patentujících institucí a jejich dat.
\newline
\indent Celkem bylo prostudováno 51 národních patentových institucí z celého světa. U institucí byla zkoumána hlavně dostupnost patentových dat (zda patentová instituce poskytuje svá data ke stažení zdarma nebo za peníze) a validita dat. Data byla vyhodnocena jako validní tehdy, když obsahovali všechny povinné atributy, kterými jsou: ID patentu, titulek patentu, datum přihlášení a existující autor. Z 51 národních patentových institucí splňovalo tyto dvě podmínky pouze 10 institucí, které poskytly necelé dva miliony validních záznamů o patentových dat.
\newline
\indent Vzhledem k získaným datům z patentových institucí byly prozkoumány možné typy databází, které by umožňovali jejich efektivní vytěžování. Při průzkumu bylo porovnáváno pouze šest nejznámějších typů databází, které by mohli ukládat patentová data. Z šesti typů dat byly nakonec vybrány dva typy databází, každý pro jiný účel. První typ, relační databáze, poslouží k~rychlému získávání statistik o patentech. Druhý typ, databáze dokumentů, poslouží k ukládání celých souborů s patentovými daty. Jako existující řešení pro relační databázi bylo vybráno MySQL, pro databázi dokumentů zase MongoDB, které se pomocí Apache Kafka propojilo s full-textovým vyhledávačem Elasticsearch. Vzhledem ke vstupním datům, která jsou uložena v několika různých strukturách, bylo dokázáno, že kombinace relační a dokumentové databáze může být velice dobré řešení. Udržování obou databází může být časově a kapacitně náročnější, ale rychlost a efektivita vyhledávání je větší než v případě použití pouze jedné databáze.
\newline
\indent Zadání práce bylo splněno ve všech bodech. Zvolená databázová řešení umožňují efektivní vytěžování patentových dat pomocí full-textového vyhledávače Elasticsearch v případě MongoDB, a pomocí \gls{SQL} dotazů v MySQL pro získávání statistik. Efektivní vytěžování bylo otestováno na třech scénářích pro MongoDB, a devíti scénářích pro MySQL. Instalace všech nástrojů a databází se provádí pomocí Dockeru.

%%%%%%%%%%%%%%%%%%%%%%%%%%%%%%%%%%%%%%%%%%%%%%%%%%%%%%%%%%
%
% Zkratky
%
%%%%%%%%%%%%%%%%%%%%%%%%%%%%%%%%%%%%%%%%%%%%%%%%%%%%%%%%%%

\printglossary[type=\acronymtype,title={Zkratky}]
 
%%%%%%%%%%%%%%%%%%%%%%%%%%%%%%%%%%%%%%%%%%%%%%%%%%%%%%%%%%
%
% LITERATURA
%
%%%%%%%%%%%%%%%%%%%%%%%%%%%%%%%%%%%%%%%%%%%%%%%%%%%%%%%%%%

\bibliographystyle{csplainnatkiv}
{\raggedright\small
\bibliography{literature/literatura}
}

%%%%%%%%%%%%%%%%%%%%%%%%%%%%%%%%%%%%%%%%%%%%%%%%%%%%%%%%%%
%
% PŘÍLOHA
%
%%%%%%%%%%%%%%%%%%%%%%%%%%%%%%%%%%%%%%%%%%%%%%%%%%%%%%%%%%
\setcounter{chapter}{0}
\renewcommand{\thechapter}{\Alph{chapter}}

%CHAPTER
\chapter{Uživatelská dokumentace}
% Postup instalace + i obrázky - po instalaci všech image obrázek z dockeru, po inicializaci MySQL v PHPMyAdmin, po inicializaci MongoDB v Mongo-express a Elasticsearch v Elasticvue
%\input{chapters/vysledny_vzhled.tex}

%%%%%%%%%%%%%%%%%%%%%%%%%%%%%%%%%%%%%%%%%%%%%%%%%%%%%%%%%%
%
% KONEC TEXTU PRÁCE
%
%%%%%%%%%%%%%%%%%%%%%%%%%%%%%%%%%%%%%%%%%%%%%%%%%%%%%%%%%%
\end{document}
