%%%%%%%%%%%%%%%%%%%%%%%%%%%%%%%%%%%%%%%%%%%%%%%%%%%%%%%%%%
%
% Vzor pro sazbu kvalifikační práce
%
% Západočeská univerzita v Plzni
% Fakulta aplikovaných věd
% Katedra informatiky a výpočetní techniky
%
% Petr Lobaz, lobaz@kiv.zcu.cz, 2016/03/14
%
%%%%%%%%%%%%%%%%%%%%%%%%%%%%%%%%%%%%%%%%%%%%%%%%%%%%%%%%%%

% Možné jazyky práce: czech, english
% Možné typy práce: BP (bakalářská), DP (diplomová)
\documentclass[czech,DP]{thesiskiv}

% Definujte údaje pro vstupní strany
%
% Jméno a příjmení; kvůli textu prohlášení určete, 
% zda jde o mužské, nebo ženské jméno.
\author{Bc. Vojtěch Danišík}
\declarationmale

% Název práce
\title{Titul práce\\nanejvýš na dvě\\až tři řádky}

% 
% Texty abstraktů (anglicky, česky)
%
\abstracttexten{The text of the abstract (in English). It contains the English translation of the thesis title and a short description of the thesis.}

\abstracttextcz{Text abstraktu (česky). Obsahuje krátkou anotaci (cca 10 řádek) v češtině. Budete ji potřebovat i při vyplňování údajů o bakalářské práci ve STAGu. Český i anglický abstrakt by měly být na stejné stránce a měly by si obsahem co možná nejvíce odpovídat (samozřejmě není možný doslovný překlad!).
}

% Na titulní stranu a do textu prohlášení se automaticky vkládá 
% aktuální rok, resp. datum. Můžete je změnit:
%\titlepageyear{2016}
%\declarationdate{1. března 2016}

% Ve zvláštních případech je možné ovlivnit i ostatní texty:
%
%\university{Západočeská univerzita v Plzni}
%\faculty{Fakulta aplikovaných věd}
%\department{Katedra informatiky a výpočetní techniky}
%\subject{Bakalářská práce}
%\titlepagetown{Plzeň}
%\declarationtown{Plzni}

%%%%%%%%%%%%%%%%%%%%%%%%%%%%%%%%%%%%%%%%%%%%%%%%%%%%%%%%%%
%
% DODATEČNÉ BALÍČKY PRO SAZBU
% Jejich užívání či neužívání záleží na libovůli autora 
% práce
%
%%%%%%%%%%%%%%%%%%%%%%%%%%%%%%%%%%%%%%%%%%%%%%%%%%%%%%%%%%

% Zařadit literaturu do obsahu
\usepackage[nottoc,notlot,notlof]{tocbibind}

% Umožňuje vkládání obrázků
\usepackage[pdftex]{graphicx}

% Odkazy v PDF jsou aktivní; navíc se automaticky vkládá
% balíček 'url', který umožňuje např. dělení slov
% uvnitř URL
\usepackage[pdftex]{hyperref}
\hypersetup{colorlinks=true,
  unicode=true,
  linkcolor=black,
  citecolor=black,
  urlcolor=black,
  bookmarksopen=true}

% Při používání citačního stylu csplainnatkiv
% (odvozen z csplainnat, http://repo.or.cz/w/csplainnat.git)
% lze snadno modifikovat vzhled citací v textu
\usepackage[numbers,sort&compress]{natbib}

%%%%%%%%%%%%%%%%%%%%%%%%%%%%%%%%%%%%%%%%%%%%%%%%%%%%%%%%%%
%
% VLASTNÍ TEXT PRÁCE
%
%%%%%%%%%%%%%%%%%%%%%%%%%%%%%%%%%%%%%%%%%%%%%%%%%%%%%%%%%%
\begin{document}
%
\maketitle
\pagestyle{empty}
\tableofcontents
\pagestyle{plain}
\addtocontents{toc}{
\protect\thispagestyle{empty}} 

\thispagestyle{empty}
\setcounter{page}{0} 

%CHAPTER
\chapter{Úvod}
 
 
%%%%%%%%%%%%%%%%%%%%%%%%%%%%%%%%%%%%%%%%%%%%%%%%%%%%%%%%%%
%
% LITERATURA
%
%%%%%%%%%%%%%%%%%%%%%%%%%%%%%%%%%%%%%%%%%%%%%%%%%%%%%%%%%%
\bibliographystyle{csplainnatkiv}
{\raggedright\small
\bibliography{literature/literatura}
}

%%%%%%%%%%%%%%%%%%%%%%%%%%%%%%%%%%%%%%%%%%%%%%%%%%%%%%%%%%
%
% PŘÍLOHA
%
%%%%%%%%%%%%%%%%%%%%%%%%%%%%%%%%%%%%%%%%%%%%%%%%%%%%%%%%%%
\setcounter{chapter}{0}
\renewcommand{\thechapter}{\Alph{chapter}}

%%CHAPTER
\chapter{Uživatelská dokumentace}
% Postup instalace + i obrázky - po instalaci všech image obrázek z dockeru, po inicializaci MySQL v PHPMyAdmin, po inicializaci MongoDB v Mongo-express a Elasticsearch v Elasticvue
%%CHAPTER
\chapter{Testovací reporty}
\label{chap:testovaci_reporty}

%SECTION
\section{Tester 1}

\textbf{Vyskytly se nějaké problémy při stahování či nahrávání dokumentu PDF?}
\newline
Ne, vše se nahrálo i stáhlo v~pořádku a včas.
\newline
\newline
\textbf{Jak byste ohodnotil vzhled dokumentu jako celek?}
\newline
Běžný formulář, který neurazí, ale ani nenadchne, každopádně je vcelku přehledný.
\newline
\newline
\textbf{Jak hodnotíte vzhled formuláře a použité prvky reprezentující jednotlivé hodnotící parametry?}
\newline
Oceňuji prostor pro delší odpovědi, ale zaškrtávací políčka jsou příliš u~sebe a je problém rozpoznat, co je vyplněno a co ne tím, že jsou jednotlivé body od sebe vcelku vzdáleny.
\newline
\newline
\textbf{Byla velikost textového pole dostatečně velká pro případné komentáře ohledně vědeckého příspěvku?}
\newline
Ano.
\newline
\newline
\textbf{Jak hodnotíte rychlost stažení a nahrání dokumentu PDF?} 
\newline
Nepozorovala jsem výraznější prodlevy.
\newline
\newline
\textbf{Byly Vámi vyplněné hodnoty správně nahrány do webového portálu?}
\newline
Bohužel nikoli. Portál nezná háčky, čárky apod. a opravuje je na nesmyslné znaky. Je schopný nahrát i nesmyslný text, např.: samé tečky apod. Pokud nahraji první verz souboru vyplněnou a druhou vyplněnou jen zčásti, soubory se spojí a vyskytuje se vždy něco z~prvního i z~druhého souboru. Možná by nebylo od věci zvýraznit opravené věci z~přehrávání.
\newline
\newline
\textbf{Děkuji za vyplnění dotazníku. Pokud máte jakékoliv další poznámky, připomínky či návrhy, uveďte je, prosím, zde:}
\newline
Nevyplněno
\newpage 

%SECTION
\section{Tester 2}

\textbf{Vyskytly se nějaké problémy při stahování či nahrávání dokumentu PDF?}
\newline
Žádné problémy nenastaly během testování.
\newline
\newline
\textbf{Jak byste ohodnotil vzhled dokumentu jako celek?}
\newline
Dokument byl přehledný a věcný.
\newline
\newline
\textbf{Jak hodnotíte vzhled formuláře a použité prvky reprezentující jednotlivé hodnotící parametry?}
\newline
Každá sekce s~výběrovými tlačítky bodového ohodnocení by mohla být více odsazena od ostatních sekcí.
\newline
\newline
\textbf{Byla velikost textového pole dostatečně velká pro případné komentáře ohledně vědeckého příspěvku?}
\newline
Textová pole byla dostatečně velká pro zadání všech informací.
\newline
\newline
\textbf{Jak hodnotíte rychlost stažení a nahrání dokumentu PDF?} 
\newline
Vše proběhlo rychle v~rámci 1-2 vteřin.
\newline
\newline
\textbf{Byly Vámi vyplněné hodnoty správně nahrány do webového portálu?}
\newline
Všechny zadané informace byly správně rozeznány a nahrány na portál včetně diakritiky a speciálních znaků.
\newline
\newline
\textbf{Děkuji za vyplnění dotazníku. Pokud máte jakékoliv další poznámky, připomínky či návrhy, uveďte je, prosím, zde:}
\newline
Nevyplněno
\newpage 

%%CHAPTER
\chapter{Vzhled PDF formuláře}\label{chap:vysledny_vzhled_formulare}
\begin{center}
\begin{minipage}[b]{0.95\textwidth}
	%\framebox{
		\includepdf[pages=1,pagecommand={},frame,width=0.95\textwidth]{pdf/vysledny_vzhled.pdf}
	%}
\end{minipage}
\end{center}
\newpage
\begin{center}
\begin{minipage}[b]{0.95\textwidth}
	%\framebox{
		\includepdf[pages=2,pagecommand={},frame,width=0.95\textwidth]{pdf/vysledny_vzhled.pdf}
	%}
\end{minipage}
\end{center}

%%%%%%%%%%%%%%%%%%%%%%%%%%%%%%%%%%%%%%%%%%%%%%%%%%%%%%%%%%
%
% KONEC TEXTU PRÁCE
%
%%%%%%%%%%%%%%%%%%%%%%%%%%%%%%%%%%%%%%%%%%%%%%%%%%%%%%%%%%
\end{document}
